
\begin{figure*} [t]
%\begin{figure*} [t]
\center
 \psfig{figure=figures/arch.eps, width=3.5in, height=6.5in, angle=90}
 \caption{\label{fig:architecture}  A modified 16-way L2 cache architecture with a 2-bit counter and a small buffer}
%\end{figure*}
\end{figure*}


%In Section 3, we argued that 10 ms is the ideal retention time for L2 cache blocks by considering both application characteristics
%and technology aspect.
We observe from Figure~\ref{fig:distribution} that on an average, approximately 50\% of the blocks will
expire after $10ms$, if no action is taken. This expiration of blocks will not only result in additional cache misses,
but would also
result in data loss, if they were dirty. In this section, we propose our architectural solutions starting with
a naive scheme of writing back
all the dirty blocks to a more sophisticated scheme, where we minimize the number of refreshes and write backs.

\subsection{{Volatile STT-RAM}}
In this design, we write back all the unrefreshed dirty blocks which become volatile after the
retention time. To identify these blocks, we maintain a counter per cache block. Each cache block
uses an {\it n}-bit counter (shown in Figure~\ref{fig:architecture}(b)(i)).
%To understand the working of the counter, let us consider an {\it n}-bit counter as shown in Figure
%~\ref{fig:architecture}(b)(i).
We assume the time between transitions (T) from one state to another
equals to the retention time divided by the number of states, where the number of states is 2$^n$. A
block starts in state {\it S$_0$} when it is first brought to the cache. After every transition time
(T), the counter of each block is incremented. When a block reaches  state {\it S$_{n-1}$}, it
indicates that it is going to expire in time T. We define this time as the
$leftover$ time and
the block in state {\it S$_{n-1}$} as  the $diminishing$ block. Increasing the value of {\it n} will
decrease the leftover time at the cost of increased overhead of checking the blocks at a finer
granularity. For example, if we use a 2-bit counter, the leftover time is $2.5ms$ and for a
3-bit counter it is $1.25ms$. A large bit counter decreases the leftover time and allows
more time for a block to stay in the cache before applying any refreshing techniques. This gives the
block more opportunity to stay in the cache. The down side is the overhead of designing and
maintaining a large bit counter.

Our experimental results show that a 2-bit counter, similar to the one used in
\cite{cache-decay-2001}, suffices to detect the expiration time of the blocks without significantly
affecting the performance. With a 2-bit counter, a block can be in one of the four states as shown in
Figure~\ref{fig:architecture} (b)(ii). A block moves from state {\it S$_0$} to state {\it S$_3$} in
steps of $2.5ms$ and at any time the block is written/invalidated, it goes back to the initial
state.  The counter bits are kept as a part of the SRAM tag array. The overhead of the 2-bit counter
is 0.4\% for one L2 cache bank. This scheme has a negative impact on the performance for two reasons:
(1) There will be  a large number of write backs to the main memory for all the dirty blocks at the
end of the retention time. (2) The expired block could have been frequently read and losing it will
incur additional read misses. We evaluate the results of this design in Section~\ref{sec:results}.



\subsection{{Revived STT-RAM Scheme}}
In order to minimize the write back overhead of the expired blocks at the end of retention time, we propose
a different technique, where we use a small buffer to hold a subset of diminishing blocks. We call this design the {\it revived STT-RAM} scheme. These dirty blocks are, thus, not written back to the main memory. They are simply written to the temporary buffer and
written back to the cache to start another fresh cycle.
Figure~\ref{fig:architecture}(a) shows the schematic diagram of the proposed scheme.
The main components of this design are a small buffer and a buffer controller.

\noindent\textbf{Buffer:} It is a per bank small storage space with a fixed number of entries made up of low-retention time STT-RAM cells.
We use these entries to temporarily store the diminishing blocks. In order to calculate the optimal MRU slots for buffering,
we collected statistics of MRU positions of diminishing blocks by running various PARSEC 2.1 and SPEC 2K6 Benchmarks on the M5 Simulator.
Figure ~\ref{fig:cdf} shows the average cumulative distribution of diminishing blocks per bank
as a function of the number of ways in a set. We observe that the number of diminishing blocks becomes stable
after first eight MRU ways. The mean number of blocks corresponding to the first eight ways is 1900
(3\% overhead over per L2 cache bank),
which is a good initial choice for the buffer size. In sensitivity analysis, we will fine tune the buffer size
to minimize buffer overflows.

\noindent\textbf{Buffer Controller:}
The buffer controller consists of a log$_2$N bit buffer overflow detector, where N is the buffer size.
%The buffer overflow detector is incremented when a diminishing block
%is copied to one of the buffer slots.
The overflow detector is first checked to see the occupancy of the buffer,
when a diminishing block is directed to the buffer and the buffer overflow detector is incremented.
The block is copied to one of the empty buffer entries along with the set and way ID, if there is
buffer space.
If the buffer is full, the dirty blocks are written back to the main memory; otherwise they are  invalidated.

\noindent\textbf{Implementation Details:} Figure ~\ref{fig:architecture} (a) shows a 16-way set
associative cache bank with the associated tag array. We show the working of our scheme using a 2-bit
counter.  One of the sets is shown in detail to clarify the details of the scheme. All the blocks in
a set are marked with their current state. Each bank is  associated with a buffer and the buffer
controller. Let us consider that the buffering scheme incorporates eight MRU slots, based on the
cumulative distribution of MRU slots shown in Figure~\ref{fig:cdf}. In Section ~\ref{sec:results}, we
will vary the number of slots to see the effects on performance. In Figure~\ref{fig:architecture}(a),
\ding{182} shows that three blocks in first eight MRU slots are diminishing and directed to the
buffer. Please note that we apply our scheme only to the diminishing (to-be-expired) blocks, which gives enough time for the scheme to be completed before actual data loss happens. \ding{183} checks the occupancy of the buffer and if it is not full, each of the diminishing
blocks is copied to one of the entries of \ding{184} along with way and set ID. Way and set ID are
again used by  \ding{183}  to copy back the blocks to the same place in the L2 cache.
\textcircled{\raisebox{-.9pt}{A}} shows the blocks which are not in MRU slots, but are diminishing.
We check these blocks in \textcircled{\raisebox{-.9pt}{B}} to see whether they are dirty or not. If
they are dirty, we write back those blocks as shown in \textcircled{\raisebox{-.9pt}{C}} and then
invalidate.  If they are not dirty, they are just invalidated. During this whole refresh process, if a read request for the cache block comes, it will be successfully completed as the data is still valid during that time. If a write/invalidate request comes, the cache blocks goes back to its original state. To make sure we don't copy the stale data
from the buffer, we do perform state check before copying back as shown in \ding{185}. Moreover, our implementation assumes the worst-case temperature of $125\,^{\circ}\mathrm{C}$ and hence there is no possibility of early expiration of block because of sudden reduction in retention time of STT-RAM cells.


%\noindent\textbf{Choosing Optimal Buffer  Size and MRU Slots:}
%In order to calculate the optimal MRU slots for buffering, we collected statistics of MRU positions of diminishing blocks
%by running various PARSEC and SPEC Benchmarks on the M5 Simulator.
%Figure ~\ref{fig:cdf} shows the average cumulative distribution of expired blocks per bank
%as a function of the number of ways in a set. We observe that the number of diminishing blocks becomes stable
%after first eight MRU ways. The mean number of blocks corresponding to the first eight ways is 1900
%(3\% overhead over per L2 cache bank),
%which is a good initial choice for the buffer size. In sensitivity analysis, we will fine tune the buffer size
%to minimize buffer overflows.


\begin{figure*} [t]
%\begin{figure*} [t]
\centering
 \psfig{figure=figures/cdf.eps, width=3in, height=1.9in}
  \caption{\label{fig:cdf} Cumulative distribution of diminishing blocks per bank with number of ways}
%\end{figure*}
\end{figure*}
