In this section, we will discuss all the experimental results associated with our design choices.



\begin{figure*} [t]
%\begin{figure*} [t]
\centering
 \psfig{figure=figures/parsec-speedup.eps, width=6.9in, height=2.5in}
 \hrule
 \caption{\label{fig:parsec-new} \scriptsize \bf Normalized speedup for PARSEC Applications }
%\end{figure*}
\end{figure*}



\noindent\textbf {Performance:} Figure \ref{fig:parsec-new} shows speedup improvements of a subset of PARSEC 
multithreaded applications along with the average (taken across 12 PARSEC applications listed in Table~\ref{}). 
All speedup numbers are normalized to S-1MB. When M-4MB design is used, the applications to the right of x264 application (including x264) are
showing speedup improvements over S-1MB. These applications are not only benefiting from 4x capacity increase but also 
because of the presence of write buffer placed on top of L2 cache. To see the benefits of write buffer,  consider fluidanimate and vips applications.
Even though they have high number of writes to the L2 cache, the writes are staggered, which help the write buffer in amleirorating increased write latency.
The applications to the left of x264 are write intensive applications. In all these applications we see degradation in speedup
because of the high write latency.  On an average, traditional 10 year STT-RAM gives 3\% speedup improvement over S-1MB.



\begin{figure*} [t]
%\begin{figure*} [t]
\centering
 \psfig{figure=figures/spec-ws.eps, width=3.9in, height=2.5in}
 \hrule
 \caption{\label{fig:spec-new} \scriptsize \bf Normalized Average Instruction Throughput(IT) and Weighted Speedup(WS) for SPEC 2006 multiprogrammed mixes. }
%\end{figure*}
\end{figure*}


Consider Volatile M-4MB(1sec) design. This design has no refreshing scheme, but since within 1 sec interval,
almost all the blocks are refreshed inherently as per application characterstics discussed in Section ~\ref{}.
We observe the benefits of this design over M-4MB in all the applications because of the reduced
write latency. Volatile M-4MB(10ms) design also 
doesn't have any refreshing scheme but the retention time of STT-RAM
cells used is 10ms, which triggers large number of write backs. 
Figure~\ref{fig:writebacks} shows number of writebacks of all the designs normalized to M-4MB. We observe that this design on an average has 21\% more 
write backs then the traditional STT-RAM design. For this reason, in vips there is about 20\% speedup degradation over M-4MB.
It is interesting to see the case of swaptions where there is slight improvement in speedup over M-4MB, although 
there is increase in number of writebacks. The reason for this improvement is on the account of the fact that, 
the majority of blocks which are not refreshed within 10ms interval, 
are not accessed in future as well leading to low number of read misses. This helps in reaping benefits
 from the reduced write latency.  

Our scheme revived M-4MB(10ms), which incorporates refreshing scheme is showing speedup improvments
in almost all applications except fluidanimate, in which all the dirty blocks are almost equally distributed among
all the ways, hence our scheme of using first eight MRU slots doesn't prevent in stopping writebacks. This
observation can also be made from Figure~\ref{fig:writebacks}, where writebacks in our scheme are also
very high. On an average our scheme is better than traditional 10yr STT-RAM by 15\% and over Volatile STT-RAM (1sec) 
by 4.5\% (In case of facesim, our scheme is better than Volatile STT-RAM (1 sec) by 22.7\%). 
Our scheme is closest to the ideal S-4M case with difference of only 4\%.

Figure~\ref{fig:spec-new} shows instruction throughput and weighted speedup for multiprogram
mixes. We observe that our scheme revived M-4MB gives 22\% improvement in instruction throughput 
over M-4MB and 11\% improvement over Volatile STT-RAM (1 sec). (In case of the mix of bzip2, gcc, lbm, hmmer, 
the improvement is 15\%). Weighted speedup improvement over M-4MB is 4\% 
and over Volatile STT-RAM(1 sec) is 2\%. 


\noindent\textbf {Energy:}
Figure~\ref{fig:energy} shows normalized leakage, total of dynamic read and write energy, and total energy.
The number of reads and writes to L2 cache are only considered for the calculation of dynamic energy. 
We observe that on an average there is 44\% improvement in total energy going from S-1MB to
M-4MB designs. The improvement is mainly because of the drastic reduction in leakage energy. 
Volatile M-4MB(1sec) leakage benefits over M-4MB correlates with the performance improvement.
On an average, this design consume more dynamic energy than M-4MB. The dynamic energy fluctuations among
different applications are on account of changes in number of read and writes. Additional write backs
triggers read misses which ultimately lead to additional writes to L2 cache. Write energies of 1sec design is more
than the 10ms designs, which makes the fluctuations depend on the number of reads/writes. 

We see 11\% energy benefits of using revived M-4MB design over Volatile-1sec and 30\% improvement over
M-4MB designs.The energy numbers of this scheme covers all the overheads of  the buffer design. 
We observe that our scheme is better in terms of both performance and energy over Volatile-4MB(1sec) and
M-4MB designs. 

\begin{figure*} [t]
\centering
\begin{tabular}{c}
\psfig{figure=figures/legend.eps, width=5.5in, height=0.15in}
\end{tabular}
\begin{tabular}{ccc}
 \psfig{figure=figures/leak-eng.eps, width=2.1in, height=2.0in} &
\psfig{figure=figures/dyn-eng.eps, width=2.1in, height=2.0in} &
\psfig{figure=figures/tot-eng.eps, width=2.1in, height=2.0in} \\
\scriptsize (a) Leakage Energy  & \scriptsize (b) Dynamic Energy & \scriptsize (c) Total Energy
\end{tabular}
 \hrule
 \caption{\scriptsize \bf Energy of Applications Normalized to that of S-1MB}
\label{fig:energy}
\end{figure*}







\subsection{Sensitivity Analysis}

\begin{figure*} [t]
%\begin{figure*} [t]
\centering
 \psfig{figure=figures/writebacks.eps, width=4.9in, height=2.5in}
 \hrule
 \caption{\label{fig:writebacks} \scriptsize \bf Number of Write backs normalized to M-4MB}
%\end{figure*}
\end{figure*}


\begin{figure*} [t]
%\begin{figure*} [t]
\centering
 \psfig{figure=figures/confi.eps, width=3.9in, height=2.5in}
 \hrule
 \caption{\label{fig:confi} \scriptsize \bf Confidence Intervals of Dead Blocks for each Way}
%\end{figure*}
\end{figure*}

\noindent\textbf {Choosing correct buffer slots}

\noindent\textbf {Number of bits for counter}
There is no observable difference in performance and energy by increase in the number of bits of the counter. 
\subsection{Scaling Effects}







