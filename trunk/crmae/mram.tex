%&latex
\documentclass[12pt,letterpaper]{article}
\usepackage{setspace}
\doublespacing
\usepackage{iccv}
\usepackage{times}
\usepackage{subfigure}
\usepackage{multirow}
\usepackage{multicol}
\usepackage{wrapfig}
\usepackage{epsfig}
\usepackage{graphicx}
\usepackage{amsmath}
\usepackage{amssymb}
\usepackage{pifont}


\newcommand{\squishlist}
{
   \begin{list}{$\bullet$}
    {
    \setlength{\itemsep}{0pt}
    \setlength{\parsep}{0pt}
    \setlength{\topsep}{3pt}
    \setlength{\partopsep}{0pt}
    \setlength{\listparindent}{-2pt}
    \setlength{\itemindent}{-5pt}
    \setlength{\leftmargin}{0.5em}
    \setlength{\labelwidth}{0em}
    \setlength{\labelsep}{0.5em}
    }
}

\newcommand{\squishend}
{
    \end{list}
}

\newcommand{\ignore}[1]{}  %definition of \ignore command

% Include other packages here, before hyperref.

% If you comment hyperref and then uncomment it, you should delete
% egpaper.aux before re-running latex.  (Or just hit 'q' on the first latex
% run, let it finish, and you should be clear).
\usepackage[pagebackref=true,breaklinks=true,letterpaper=true,colorlinks,bookmarks=false]{hyperref}


% \iccvfinalcopy % *** Uncomment this line for the final submission

\def\iccvPaperID{761} % *** Enter the MICRO Paper ID here
\def\httilde{\mbox{\tt\raisebox{-.5ex}{\symbol{126}}}}

% Pages are numbered in submission mode, and unnumbered in camera-ready
\ificcvfinal\pagestyle{empty}\fi
\begin{document}

%%%%%%%%% TITLE
\title{Cache Revive: Tuning Retention times of STT-RAM Caches for Enhanced
Performance in CMPs.}

\author{First Author\\
Institution1\\
Institution1 address\\
{\tt\small firstauthor@i1.org}
% For a paper whose authors are all at the same institution,
% omit the following lines up until the closing ``}''.
% Additional authors and addresses can be added with ``\and'',
% just like the second author.
% To save space, use either the email address or home page, not both
\and
Second Author\\
Institution2\\
First line of institution2 address\\
{\small\url{http://www.author.org/~second}}
}

\maketitle
% \thispagestyle{empty}

%%%%%%%%% ABSTRACT
%-------------------------------------------------------------------------
\begin{abstract}
\singlespacing

Spin-Transfer Torque RAM (STT-RAM) is an emerging non-volatile memory (NVM)
technology that has the potential to replace the conventional
on-chip SRAM caches for designing a more efficient memory hierarchy for
future multicore architectures.
%While the high density, low leakage and high endurance
%of STT-RAM are attractive compared to SRAM,
However, it's long write latency and high dynamic
write energy are major obstacles for being competitive with the SRAM-based cache hierarchy.
On the other hand, the non-volatility feature with years of data retention time for STT-RAM technology is not necessary for the usage of STT-RAM as on-chip cache, since the life time of cache data are usually within us or ms. Consequently,  we exploit such observation for designing an efficient L2 cache
architecture, and propose to trade off the non-volatility (data retention time)
for better write performance/energy in STT-RAM cache design. The paper addresses several critical design issues such as how do we decide a suitable retention time for last level cache,
what is the relationship between retention time and write latency,
and how do we architect the cache hierarchy with a volatile STT-RAM.
%Through an extensive execution driven analysis of the inter-write time of several PARSEC and SPEC 2006 benchmarks, we
%observe that retention time in the order of 10-40 ms is a good design point to handle most of the
%writes.
We study two
data-retention relaxation cases, one with data retention time of 1 second, which satisfies the lifetime requirement of typical cache blocks; and the other one with data retention time of 1ms, which is a more aggressive design for better performance/energy gains but a data refreshing mechanism is needed.
In the aggressive data retention time relaxation design, for the rest of the cache blocks that have a higher inter-write
time than the STT-RAM retention time, we propose an architectural solution to identify these blocks
with a per block 2 bit counter, temporarily save a limited number of MRU blocks in a buffer,
and write-back the rest of the dirty blocks to avoid any data loss.
%are not in first seven MRU slots.
%The blocks in these slots are copied to
%a per bank small buffer and again copied back to the respective slots.
Our experiments with 4 and 8-core architectures with an SRAM-based L1 cache and STT-RAM-based L2 cache
indicate that not only we can eliminate the high write overhead of an NVM STT-RAM, but can provide
on an average 10-12\% improvement in IPC compared to the traditional SRAM-based
design, while reducing the energy consumption significantly


\end{abstract}
%-------------------------------------------------------------------------
%%%%%%%%% BODY TEXT
%-------------------------------------------------------------------------
\section{Introduction} \label{sec:intro}

Designing an efficient memory hierarchy for multicore architectures is a critical but challenging
problem. As the number of cores on a chip increase with technology scaling, the demand on the on-chip
memory would increase significantly, further worsening the memory wall problem~\cite{BurgerGK96}. The
memory wall problem is critical both from the performance (memory density) and power perspectives.
Thus novel technology, circuit and architectural techniques are currently being explored to address
the memory wall problem for many core systems.


% Dr. Yuan Changes to first two lines.
%Spin-Transfer Torque RAM (STT-RAM) is a promising memory technology that has the potential to replace the conventional %on-chip SRAM
%caches,  because of its higher density, competitive read times and lower leakage power consumption
%compared to SRAM.


Spin-Transfer Torque RAM (STT-RAM) is a promising memory technology that delivers on many aspects
desirable of an universal memory. It has the potential to replace the conventional on-chip SRAM
caches because of its higher density, competitive read times, and lower leakage power consumption
compared to static-RAM (SRAM). However, the high write latencies and write energy are key drawbacks
of this technology for providing competitive or better performance compared to the SRAM-based cache
hierarchy. Consequently, recent efforts have focused on masking the effects of high write latencies
and write energy at the architectural level~\cite{mram-energy-reduction,gsun-hpca,}. In contrast to
these architectural approaches, a recent  work explored the {\it feasibility} of relaxing STT-RAM
data retention times to reduce both write latencies and write energy~\cite{STTRAM:HPCA11}. This
adaptable feature of tuning the data retention time can be exploited in several dimensions. The focus
of this paper is to tune this data retention time to closely match the required lifetime of the last
level cache blocks to achieve significant performance and energy gains. In this context, the paper
addresses several design issues such as how to decide an appropriate retention time for the last
level caches, what is the relationship between retention time and write latency, and how do we
architect the cache hierarchy with a volatile STT-RAM.

The non-volatile nature and non-destructive read ability of STT-RAM provides a key difference with
regard to traditional on-chip cache design with SRAM technology. However, as our analysis will show,
for many applications, it is sufficient if the data stored in the last level of a cache hierarchy
remains valid for a few tens of milliseconds. Consequently, the duration of data retention in STT-RAM
is an obvious candidate for device optimization for the cache design. We, therefore, conduct an
application-driven study to analyze the inter-write times of the L2 cache blocks to decide a suitable
data retention time. Although lifetime analysis of cache lines has been the conducted earlier to
improve performance and reduce power consumption~\cite{cache-decay-2001,3t1d-cache}, we revisit this
topic with a different intention - correlating STT-RAM data retention time to cache life time. An
extensive analysis of PARSEC and SPEC 2006 benchmarks using the M5 simulator~\cite{M5} indicates that
the average inter-write times for most of the L2 cache blocks is close to 10ms, and thus, we advocate
our STT-RAM design with this retention time.

We conduct a detailed device level analysis of the STT-RAM cells to analyze the write current versus
write pulse width tradeoffs, cell area analysis, and retention time stability analysis to capture the
relationship between area, read/write latency and leakage power as a function of the retention time.
Our observations, in contrast to the results reported in~\cite{STTRAM:HPCA11} indicate that retention
times in the range of milliseconds ({\it ms}) are probably more achievable than in the microseconds
({\it $\mu$s}) range. Thus, using this {\it ms} range retention time model, we then propose effective
architectural techniques to avoid any data loss due the volatile STT-RAM-based cache hierarchy.

A key challenge in determining a suitable data retention times for the STT-RAM is to balance the
reduced write latency of cells with lower retention time against the overhead for data refresh or
write back of cache lines with longer lifetimes. In this paper, we compare 3 different STT-RAM based
cache designs: (1) STT-RAM cache without retention time relaxation (10+ years of data retention
time); (2) STT-RAM cache with retention time of 1 second, which is long enough for the lifetime of
majority of the cache lines and therefore no refreshing overhead is incurred; (3)STT-RAM cache with
retention time of 10ms, which is a more aggressive design with better performance/energy gain but a
data refreshing technique is needed for correct operations since cache lines that have lifetimes
exceeding 10ms are likely to loose data. Thus, we propose simple extensions to the L2 cache design
for avoiding any data loss. This include a simple 2-bit counter (similar to one proposed
in~\cite{cache-decay-2001}) to keep track of the lifetime of all the cache blocks and a small buffer
to temporarily store the blocks whose time has exceeded the retention time. We conduct
execution-driven analysis of our proposed techniques using the M5 simulator and a suite of PARSEC and
SPEC 2006 benchmarks. The main contributions of this work are the following:

\noindent\textbf{Detailed characterization of STT-RAM volatile property:} We present a detailed
device characterization of data retention tunability in STT-RAM Cells providing insight to the
underlying principles enabling these tradeoffs. We believe the design in~\cite{STTRAM:HPCA11} is very
aggressive and may not be feasible considering the state-of-the-art in device technology. Moreover,
as our analysis shows, a very aggressive {\it $\mu$s} level retention time is unsuitable for last
level cache block.
%On the other hand, if the microsecond level retention time is feasible, then our proposed
%architectural solution would be even more beneficial.

\noindent\textbf{An application-driven study to determine retention time:}
We analyze the time between writes or replacements to a cache line for
various multi-threaded and multi-programmed workloads. Our characterization
augments the prior body of work that analyzes cache lifetimes mainly in
single processor and single program configurations. Based on the L2 cache
behavior, we propose to design STT-RAMs with retention time in the range of 10ms.

\noindent\textbf{Architectural solution to handle STT-RAM volatility:} We present a simple buffering
mechanism to ensure the integrity of programs given the volatile nature of our tuned STT-RAM cells.
Experimental results with PARSEC and SPEC benchmarks on a four-core and eight-core multi-core platform
compared to a base case 1MB SRAM per core and the ideal 4MB SRAM per core indicate that the proposed
solution is attractive both from performance and power perspectives. On an average, we find XX\% IPC
improvement with PARSEC benchmarks, XX\%/XX\% instruction throughput/weighted speedup improvement
with SPEC 2006 benchmarks, and an average energy saving of XX\% across our entire application suite.

%\noindent\textbf{Comparison with prior architectural proposals:} We show that our scheme is better by
%XX\% when compared to a recent write latency hiding scheme proposed in~\cite{}. Additionally, all our
%results are with a write-buffer ...


The rest of this paper is as follows: In Section~\ref{sec:design}, we discuss the volatile STT-RAM design to
parameterize the retention time and write latency behavior. Section~\ref{sec:motivation} presents a retention time
estimation study from application perspective. Following this, the design of a volatile STT-RAM-based last
level cache architecture is given in Section~\ref{sec:implementation}; the experimental platform details in Section~\ref{sec:evaluation},
results in Section~\ref{sec:results}, and description of related work in Section~\ref{sec:prior_work}.
The last section provides concluding remarks.


%- Finally, we show that our combined device-architecture life time
%tuning approach is better than recent efforts that attempt to address the
%long write latencies of STT-RAM.

%cache lifetimes performed for a multi-threaded workload demonstrates
%that a significant fraction of L2 cache lines can operate correctly
%without any additional support when the STT-RAM retention times are of
%the order of 50ms. However, architectural support is required to ensure
%that correct program state is maintained for the rest of the cache lines
%that have lifetimes exceeding 50ms. While a simple
%DRAM-style refresh has been proposed in~\cite{STTRAM:HPCA11} to ensure
%correctness, it is possible to avoid many of these refresh by pursuing a
%life-time aware refresh strategy.

%-------------------------------------------------------------------------
\section{STT-RAM Design} \label{sec:design}
In this section, we first discuss STT-RAM preliminaries. This is followed by a detailed discussion of
STT-RAM models which we have developed to guide us in our exploration of suitable STT-RAM device for
LLC. Based on this, we show three stable/feasible STT-RAM cell designs
suitable for LLC in multi-core architectures.
%\begin{wrapfigure}{l}{0.50\textwidth}
%%\begin{figure*} [t]
%\centering
% \psfig{figure=figures/stt-cell.eps, width=1.6in, height=0.4\textwidth, angle=-90}
% \caption{\label{fig:mram_cell} (a) Structural view of an STT-RAM Cache Cell (b) Anti Space Parallel High Resistance, Indicating ``0" state (c) Parallel Low Resistance, Indicating ``1" state}
%%\end{figure*}
%\end{wrapfigure}
\subsection{Preliminary on STT-RAM}
\begin{figure*} [t]
\centering
\begin{minipage}{0.575\textwidth}
\centering
 \psfig{figure=figures/stt-cell.eps, width=1.5in, height=0.85\textwidth, angle=-90}
 \caption{\label{fig:mram_cell} (a) Structural view of an STT-RAM Cache Cell
 (b) Anti-Parallel High Resistance, Indicating ``0" state (c) Parallel Low Resistance, Indicating ``1" state}
\end{minipage}
\hfill
\begin{minipage}{0.375\textwidth}
\centering
 \psfig{figure=figures/IcWt.eps, width=0.85\textwidth, height=1.5in}
 \caption{\label{fig:IcWt} Demonstration of three switching phases:
 thermal activation, dynamic reversal and precessional switching }
\end{minipage}
\end{figure*}

STT-RAM uses Magnetic Tunnel Junction~(MTJ) as the memory storage and leverages the difference in
magnetic directions to represent a memory bit (``0"/``1" state). As shown in
Figure~\ref{fig:mram_cell}, an MTJ contains two ferromagnetic layers. One ferromagnetic layer has a
fixed magnetization direction and it is called the reference layer. The second layer's magnetic
direction can be changed by passing a write current, and, thus it is called the free layer. The
relative magnetization direction of two ferromagnetic layers determines the resistance of MTJ.  If
two ferromagnetic layers have different directions, the resistance of MTJ is high, indicating a ``0"
state; if two layers have the same directions, the resistance of MTJ is low, indicating a ``1"
state. The current amplitude required to reverse the
direction of the free ferromagnetic layer is determined by the size and the aspect ratio of MTJ, and
the write pulse duration~\cite{STTRAM:JAP07, STTRAM:Qualcomm09}.

\subsection{Write current Vs. Write pulse width trade-off} \label{subsec:ict}

%\begin{wrapfigure}{l}{0.40\textwidth}
%\centering
% \psfig{figure=figures/IcWt.eps, width=0.40\textwidth, height=1.45in}
% \caption{\label{fig:IcWt} Demonstration of three switching phases: thermal activation, dynamic reversal and precessional switching }
%\end{wrapfigure}

The current amplitude required to reverse the direction of the free ferromagnetic layer is determined
by many factors like material property, device geometry and most importantly the write pulse
duration. Generally, the longer the write pulse is applied, the lesser the switching current is
needed to switch the MTJ state. Three distinct switching modes were identified in~\cite{STTRAM:JAP07}
according to the operating range of switching pulse width $\tau$: thermal activation ($\tau>20ns$),
precessional switching ($\tau<3ns$) and dynamic reversal ($3ns<\tau<20ns$).

The relationship between switching current density, $J_{c}$, and write pulse width $\tau$ in the
three operating ranges was characterized by an analytical model in~\cite{STTRAM:IEDM09}. The
equations are listed as follows:

%\begin{eqnarray}
 {
 \small{
\noindent $J_{c,TA}(\tau) = J_{c0}\{1- (\frac{k_{B}T}{E_{b}})ln(\frac{\tau}{\tau_{0}})\}$
\hspace{1mm} \textbf{(1)} \hspace{1mm} $J_{c,PS}(\tau) = J_{c0}+ \frac{C}{\tau^{\gamma}}$
\hspace{1mm} \textbf{(2)} \hspace{1mm} $J_{c,DR}(\tau) =
\frac{J_{c,TA}(\tau)+J_{c,PS}(\tau)e^{-k(\tau - \tau_{c})}}{1+e^{-k(\tau - \tau_{c})}}$ \hspace{1mm}
\textbf{(3)}
 }
 }
%\end{eqnarray}

where, $J_{c,TA}$, $J_{c,PS}$, $J_{c,DR}$ are the switching current densities for thermal activation,
precessional switching and dynamic reversal respectively. $J_{c0}$ is the critical switching current
density, $k_{B}$ is the Boltzmann constant, $T$ is the temperature, $E_{b}$ is the thermal barrier,
and $\tau_{0}$ is inverse of the attempt frequency. $C$, $\gamma$, $k$, and $\tau_{c}$ are fitting
constants. Based on Figure~\ref{fig:IcWt} and analysis of the analytical model, we found vastly
different switching characteristics in the three switching modes. For example, in the thermal
activation mode, the required switching current increases very slowly even if we decrease the write
pulse width by orders of magnitude, and thus, short write pulse width is more favorable in this
region because reducing write pulse can reduce both write latency and energy without much penalty on
read latency and energy. In the precessional switching mode, write current goes up rapidly if we
further reduce the write pulse width, therefore, minimum write energy of the MTJ is achieved at some
particular write pulse width in this region. Based on this analysis, this paper will focus on the
exploration of write pulse width in precessional switching and dynamic reversal to optimize for
different design goals.

\subsection{STT-RAM Modeling}

Before simulating the performance characteristics of an STT-RAM cache, we first introduce models to
determine the area of a STT-RAM cell. Area of each STT-RAM cell would determine the area of a cache
bank composed of these cells and in turn influence the read/write latency of the bank. As shown
earlier in Figure~\ref{fig:mram_cell}, each 1T1J STT-RAM cell is composed of an NMOS and one MTJ. The
NMOS access device is connected in series with the MTJ. The size of NMOS is constrained by both SET
and RESET current, which are inversely proportional to the writing pulse width. In order to estimate
the current driving ability of MOSFET devices, a small test circuit using HSPICE with PTM 45nm HP
model~\cite{PTM} is simulated. The driving current is obtained by assuming typical TMR (120\%) and
Low Resistance State(LRS) ($3k\Omega$) value~\cite{STTRAM:Qualcomm09} and wordline voltage to be 1.5V (the optimal value is
extracted from~\cite{STTRAM:Gatech10}). Further, we oversize the access transistor width to guarantee
enough write current is provided to MTJ using the methodology discussed in~\cite{STTRAM:RPI10}. To
achieve high cell density, we model the STT-RAM cell area by referring to DRAM design
rules~\cite{DRAM:6F2}.  As a result, the cell size of a STT-RAM cell is given as:

%\begin{equation}
 {
 \small{
 \hspace{55mm} $\mathrm{Area}_{\mathrm{cell}}={3\left(W/L+1\right)}(F^2)$ \hspace{3mm} \textbf{(4)}
 }
 }
%\end{equation}

where, $W$ and $L$ are the channel width and length of the access NMOS transistor respectively.

\subsection{Impact of Retention Time on MTJ Characterstics} \label{subsec:retention}



The retention time of a MTJ is largely determined by the \textit{thermal stability} of the MTJ. The relation
between retention time and thermal barrier is shown in Figure~\ref{fig:retention}, which can be
modeled as $t=C\times e^{k\Delta}$, where $t$ is the retention time and $\Delta$ is the thermal
barrier, while $C$ and $k$ are fitting constants. Thermal stability of the free layer in an MTJ has impacts not
only on the retention time of STT-RAM memory cell, but also on the write current. It was found
in~\cite{PMTJ:Toshiba08} that the switching current of MTJ decreases as thermal barrier is reduced. Here, we combine this observation with the write current versus write time trade-off described in Section~\ref{subsec:ict}, which essentially means that once the thermal barrier of a MTJ is lowered, we are able to achieve faster write speed or/and smaller write current/energy. The baseline MTJ in our study is a $2F^2$ in-plane MTJ with a thermal barrier of $72k_{B}T$, where $k_{B}$ is the Boltzman constant and $T$ is the temperature. The non-volatile MTJ can hold data more than 10 years under a worst-case temperature of $125\,^{\circ}\mathrm{C}$. Unlike~\cite{STTRAM:HPCA11}, the optimized $2F^2$ cell does not have much room for only reducing planar area. Hence, reduction in thermal barrier was obtained by decreasing the thickness of free layer and lowering the saturation magnetization. For example, we are able to get two volatile MTJs with reduced thermal barriers of $46k_{B}T$ and $40k_{B}T$, which are corresponding for retention times of 1 second and 10 millisecond under $125\,^{\circ}\mathrm{C}$ separately. Raw experimental data were obtained from our device collaborator and we further did curve fitting by using the equations (1)-(3). The results are plotted in Figure~\ref{fig:currentVStime}. Next we are going to show how the volatile MTJs are optimized for both latency and energy. The $10ns$ write pulse width is used as the practical operating point A($10ns$, $114\mu A$) in our baseline MTJ similar to previous work~\cite{CACTI:DAC08:Dong}. Simply by fixing the write pulse width at $10ns$, we are able to lower the write currents of the two volatile MTJs which operate at B($10ns$, $73\mu A$) and C($10ns$, $40\mu A$). Then we apply the write current versus write time trade-off on the operating points of the volatile MTJs to reduce the write latency. Particularly, we operate the MTJ with 1sec-retention at point B'($5ns$, $82\mu A$) and the write current is 20\% lower than that of baseline A. Moreover, we operate the MTJ with 10ms-retention at C'($2ns$, $61\mu A$) and the write current is 20\% lower than that of B'.

%This challenge is further compounded by device variations, where data retention times follows a
%distribution.

\subsection{STT-RAM Cache Simulation Setup}

\begin{figure*} [t]
\centering
\begin{minipage}{0.575\textwidth}
\centering
 \psfig{figure=figures/RT_TB.eps, width=0.65\textwidth, height=1.7in}
 \caption{\label{fig:retention} MTJ thermal stability requirement for different retention times}
\end{minipage}
\hfill
\begin{minipage}{0.375\textwidth}
\centering
 \psfig{figure=figures/i_vs_t.eps, width=0.85\textwidth, height=1.7in}
 \caption{\label{fig:currentVStime} Write current versus write pulse width for three MTJs with $10years$, $1sec$, and $10ms$ retention at $125\,^{\circ}\mathrm{C}$ }
\end{minipage}
\end{figure*}


% \begin{figure*} [t]
% \centering
% \centering
 % \psfig{figure=figures/i_vs_t.eps, width=3.4in, height=1.9in}
 % \caption{\label{fig:currentVStime} Write current versus write pulse width for three MTJs with $10years$, $1sec$, and $10ms$ retention at $125\,^{\circ}\mathrm{C}$}
% \end{figure*}

\begin{table*}[t]
 \scriptsize
  \centering
  \caption{16-way L2 Cache Simulation Results}
  \label{allcaches}
  \begin{tabular}{| c | c | c | c | c | c | c | c |}
    \hline\hline
    \multirow{2}{*}{} & & Area  & Read Latency & Write Latency & Read Energy & Write Energy & Leakage Power\\
  & & ($mm^2$) & ($ns$) & ($ns$) & ($nJ$) & ($nJ$) & ($mW$) \\
    \hline\hline
    \multicolumn{2}{|c|}{$1MB$ SRAM} & $2.612$ & $1.012$ & $1.012$ & $0.578$ & $0.578$ & $4542$ \\
    \hline
    \multirow{3}{*}{$4MB$ STT-RAM} & $t=10yr$ & $3.003$ & $0.998$ & $10.61$ & $1.035$ & $1.066$ & $2524$ \\
%   \cline{2-2}\cline{3-8}
    & {$t=1s$} & $2.904$ & $0.973$ & $5.571$ & $1.015$ & $1.036$ & $2235$ \\
%   \cline{2-2}\cline{3-8}
    & {$t=10ms$} & $2.901$ & $0.959$ & $2.598$ & $1.002$ & $1.028$ & $2227$ \\
    \hline\hline
  \end{tabular}
\end{table*}

Based on the above device level characterizations and analytical models, we simulate SRAM-based
caches and STT-RAM-based caches with a tool called NVsim~\cite{CACTI:PCRAMsim}. NVsim which is a
circuit-level performance, energy, and area simulator based on CACTI for emerging non-volatile
memories. We integrated all the models described in the previous sub-sections in NVsim.
Table~\ref{allcaches} shows the architectural parameters based on our STT-RAM models. It shows the
read, write times and energy numbers of three stable operating points A, B', and C' for MTJs with different retention times. We find that a 4MB NVM STT-RAM cache occupies similar chip area as 1MB SRAM. This is consistent with previous work~\cite{CACTI:DAC08:Dong}. For the leakage simulation, we didn't apply any power gating techniques for the cache banks. The results show that STT-RAM consume almost half leakage power of SRAM. That is basically because half of STT-RAM die area is occupied by peripheral circuitry which means half of the chip are leaky.

By relaxing retention time of STT-RAM with lower thermal barrier, the STT-RAM cache can have smaller
area, lower write latency and less leaky peripheral circuitry. However, since retention time is
exponentially related with thermal barrier, and thermal barrier is highly sensitive to process
variation and temperature, the benefit of decreasing write latency by relaxing the retention time in
the same order of magnitude is so small that it can be easily offset by slight variation in device
geometry or environment temperature. For example, the difference of thermal barrier for $10ms$ and
$50ms$ retention MTJ is only about 5\%.
%The simulation results are listed in Table~\ref{allcaches}
%which shows the read, write times and energy numbers for three different retention times. The SRAM
%numbers are also given in the table for reference.
Based on this analysis, we propose $10years$, $1sec$ and $10ms$ STT-RAM devices for our
LLC design. In later sections, we will analyse these designs and will propose that $10ms$ is ideal
both from device as well as application side.

%-------------------------------------------------------------------------
\section{An Application-driven Approach to Determining Retention Time} \label{sec:motivation}



In order to utilize the volatile STT-RAM as the last level cache in designing an effective cache hieararchy,
we need to know what should be the ideal/feasible retention time. Ideally, the STT-RAM write  latency
should be competitive to SRAM  latency and the cache retention time should be  high.
However, as discussed in the following section, since the write latency is inversely propositonal to the
retention time, we need to find a feasible tradeoff based on the STT-RAM device characteristics.
Thus, we first attempt to decide an ideal retention time by analyzing the characteristics of a last level
cache in a multiprogrammed environment. The idea is to understand the distribution of the inter-write 
interval and thus the average inter-write time to a last level cache and use this time as the STT-RAM
retention time.
This section describes our application-driven study to estimate the retention time.



\subsection{Relating Application Characteristics to Retention Time}

\begin{figure*} [t]
\centering
\begin{tabular}{cc}
 \psfig{figure=figures/parsec-hist.eps, width=3.4in, height=2.0in} &
\psfig{figure=figures/spec-hist.eps, width=3.4in, height=2.0in} \\
\scriptsize (a) PARSEC  & \scriptsize (b) SPEC 2006
\end{tabular}
 \hrule
 \caption{\scriptsize \bf Distribution of Blocks Showing Different Revival Times}
\label{fig:distribution}
\end{figure*}

Application characterization gives the basis for evaluating the impact of low retention time STT-RAM caches on the overall system performance. In order to do this characterization, the first step is to  find the ideal time for which the cache block should retain the data.  We see that, cache block is only refreshed when the block is written. Since we are using low retention time STT-RAM for L2 cache, we record intervals between two successive writes (refreshes) to the same L2 cache block. We define this interval to be {\it revival time}. While collecting these results, we ensure that if a block gets invalidated in between two consecutive writes, we don't consider the time in between the invalidation and the next write. Previous works ~\cite{brooks} do similar type of revival time analysis but for L1 cache. Figure~\ref{fig:distribution} shows the distribution of L2 cache blocks having different revival time intervals. These results are obtained by running multi-threaded (PARSEC~\cite{PARSEC}) and multi-programmed (SPEC 2006~\cite{SPEC}) applications on M5 Simulator~\cite{M5} which models a 4GHz processor consisting of 4 cores, with 4MB STT-RAM L2 cache. Table~\ref{table:configuration} contains additional details of the system configuration. Figure~\ref{fig:distribution} (a) and (b) shows results of three PARSEC and SPEC benchmarks along with the averages across 12 PARSEC  and 14 SPEC benchmarks respectively. We observe from the figure that, on an average, approximately 50\% of cache blocks get refreshed within 10 ms. About 20\% of blocks remain in the cache for more than 40 ms and rest of the blocks have intermediate revival times. We conclude that blocks which stay longer than the retention time in the cache without being refreshed are going to affect the application performance the most. The opportunities lie with the computer architect to design micro-architecture schemes to make the unrefreshed block percentage as low as possible. This distribution also gives us the basis on which we can choose the optimal retention time. Reducing the retention time too much will make the cache too volatile leading to degraded performance. In the next subsection, we will see how increasing the retention time, has negative impacts on write latency. 



\subsection{Low Retention STT-RAM Characteristics}
Table~\ref{table:rt-wt} shows that there is significant reduction in write latency with reduction in retention time. Section 4 explains how these numbers originate. We want to clarify from device fabrication perspective that, these retention times are the most stable designs possible. As we lower the retention times of these STT-RAM cells in the range of {\it ms} it becomes much harder to precisely mark a STT-RAM cell with a fixed retention time. For the sake of correctness and preciseness we discuss these designs only in the paper. Later in the Section 7, it will be clear, that our design assumptions have no affect on the generality of the results.

To analyze the tradeoffs between retention time and overall system performance, lets consider an utopian cache with 10 year retention time having minimum write latency and energy.  To bridge the gap between current and utopian cache, we need to reap the benefits of both: application characteristics and emerging device technology. From application side, it is best to choose a retention time which minimizes the unrefreshed blocks and from technology side it is ideal to choose the STT-RAM with minimum write latency and energy. We choose 10 ms retention time as optimal retention time which balances both the sides. In Section 7, we do a sensitivity analysis by choosing retention times 100 ms, 500 ms and 1sec. In Section 5, we propose micro-architecture techniques to deal with blocks having revival time greater than 10ms. 

\begin{table*}[t]
  \centering
  \caption{Retention and Write Latencies for STT-RAM L2 Cache}
  \label{table:rt-wt}
  \begin{tabular}{| c | c | c | c | c | c |}
  	\hline
	 Retention Time & 10years & 1sec & 500ms & 100ms & 10ms \\
	\hline
	Write Latency @4GHz & 44 cycles & 24 cycles & 20 cycles & 16 cycles & 12 cycles \\
	\hline
  \end{tabular}
\end{table*}



 
  



%-------------------------------------------------------------------------
\section{Architecting Volatile STT-RAM} \label{sec:implementation}

\begin{figure*} [t]
%\begin{figure*} [t]
\center
 \psfig{figure=figures/arch.eps, width=4.5in, height=3.5in}
 \hrule
 \caption{\label{fig:architecture} \scriptsize \bf A modified 16-way L2 cache architecture with a 2-bit counter and a small buffer}
%\end{figure*}
\end{figure*}


%In Section 3, we argued that 10 ms is the ideal retention time for L2 cache blocks by considering both application characteristics
%and technology aspect.
We observe from figure~\ref{fig:distribution} that on an average, approximately 50\% blocks will 
expire after 10 ms, if no action is taken. This expiration of blocks will not only result in additional cache misses
but also would
result in data loss, if they were dirty. In this section, we propose our architectural solutions starting with
a naive scheme of writing back
all the dirty blocks to a more sophisticated scheme, where we minimize the number of refreshes and write backs. 

\subsection{{Volatile STT-RAM}}
In this naive design, we write back all the unfreshed dirty blocks which become volatile after the retention time.
To identify these blocks, 
we maintain a counter per cache block.  To understand the working of the counter, let us consider an {\it n} bit
counter. We assume the time between transitions (T) from one state to another equals to the retention time
divided by the number of states, where the number of states is 2$^n$ .
A block strats in state {\it S$_0$} when it is first brought to the cache. After every transition time (T),
the counter of each block is incremented.
When a block reaches  state {\it S$_{n-1}$}, it indicates that it is going to expire in time T. 
We define this time as the {\it leftover time} and the block in state {\it S$_{n-1}$} as  the diminishing block. 
Increasing the value of {\it n}, will decrease the leftover time at the cost of increased overhead of checking 
the blocks at a finer granularity. 
For example, if we use a 2-bit counter, the leftover time is 2.5 ms and for a 3-bit it is 1.25ms.
A large counter decreases the {\it leftover time} and allows more time for a block to stay in the cache
before applying any refreshing techniques. The down side is the overhead of designing and maintaing a large counter. 

Our experimental results show that a 2-bit counter, similar to the one used in \cite{marget},
is sufficient enough to detect the expiration time of the blocks without significantly aafecting the performance.
With a 2-bit counter a block can be in one of the four states as shown in figure~\ref{fig:architecture} (b).
A block moves from state {\it S$_0$} to state {\it S$_3$} in steps on 2.5ms and any time the block is refreshed, it 
goes back to the initial state.  The Counter bits are kept as a part of the SRAM tag array. 
The overhead of the 2-bit counter is 0.4\% for one L2 cache bank.  

This scheme has negative impact on the performance for two reasons: (1) There will be  a large number of write
backs to the main memory for all the dirty blocks at the end of the retention time. 
(2) The expired block could have been frequently read and losing it will incur additional read misses.
We evaluate the results of this design in Section 7.

% where to include gray encoding style?

\subsection{{Revived STT-RAM Scheme}}
In order to minimize the write back overhead of the expired blocks at the end of retention time, we propose 
a different technique, where we use a small buffer to hold a subset of expired blocks at the end of the retention
time. We call this design the {\it revived STT-RAM} scheme. These dirty blocks are thus not written back to the main memory. They are simply written to the temporary buffer and
written back to the cache to start another freash cycle. 
Figure~\ref{fig:architecture} a) shows the schematic diagram of the proposed scheme.
The main components of this design are a small buffer and a buffer controller.

\noindent\textbf{Buffer:}
It is a per bank small storage space with a fixed number of entries made up of low-retention time STT-RAM cells.
We use these entries to temporarily store the diminished blocks. We estimate the optimal buffer size later in 
the section.

\noindent\textbf{Buffer Controller:}
The buffer controller consists of a log$_2$N bit buffer overflow detector, where N is the buffer size.
The buffer overflow detector is incremented when a diminishing block
is copied to one of the buffer slots. The overflow detector is first checked to see the occupancy of the buffer,
when a diminishing block is directed to the buffer. 
The block is copied to one of the empty buffer entries along with the set and way id, if there is 
buffer space. 
If the buffer is full, the dirty blocks are written back to the main memory; otherwise they are  invalidated. 

\noindent\textbf{Implementation Details:}
Figure ~\ref{fig:architecture} (a) shows a 16-way set associative cache bank with the associated tag array.
Counter bits are also placed in tag array. We show the working of
our scheme using a 2-bit counter.  One of the sets, is shown in detail to clarify the details of the scheme.
All the blocks in a set are marked with their current state. Each bank is  associated with a buffer and the buffer controller.
Let us consider that we are using the buffering scheme for eight MRU slots. Later in 
this section, we will justify this decision. In Section 7, we will vary the number of slots to see the effects on performance. 
In Figure  ~\ref{fig:architecture} (a), \ding{182} shows that three blocks in first eight MRU slots are diminishing and directed to the buffer. \ding{183} checks
the occupancy of the buffer and if it is not full, each of the diminishing blocks is copied to one of the entries
of \ding{184} along with way and set id. Way and set id are again used by the \ding{183}  to copy back the blocks to
the same place in the L2 cache. \textcircled{\raisebox{-.9pt}{A}} shows the blocks which are not in MRU slots,
but are diminishing. We check these blocks in \textcircled{\raisebox{-.9pt}{B}} to see whether they are dirty or not.
If they are dirty, we write back those blocks as shown in \textcircled{\raisebox{-.9pt}{C}}.  If they are not dirty, they are invalidated.

\noindent\textbf{Choosing Optimal Buffer  Size and MRU Slots:}
In order to calculate the optimal MRU slots for buffering, we collected statistics of MRU positions of diminishing blocks
by running various PARSEC and SPEC Benchmarks on the M5 Simulator. 
Figure ~\ref{fig:cdf} shows the average cumulative distribution of expired blocks per bank 
as a function of the number of ways in a set. We observe that the number of diminished blocks becomes stable
after first eight MRU ways. The mean number of blocks corresponding to the first eight ways is 2048
(3.16\% overhead over per L2 cache bank),
which is a good initial choice for the buffer size. In sensitivity analysisi, we will fine tune the buffer size
to minimize buffer overflows. 


\begin{figure*} [t]
%\begin{figure*} [t]
\centering
 \psfig{figure=figures/cdf.eps, width=6.9in, height=2.5in}
 \hrule
 \caption{\label{fig:cdf} \scriptsize \bf Cumulative Distribution of Dead Blocks per Bank with number of ways.}
%\end{figure*}
\end{figure*}



%-------------------------------------------------------------------------
\section{Experimental Evaluation} \label{sec:evaluation}
\begin {table}[t]
 \centering
  \singlespacing
 \caption{{Application characteristics}: {\it Read\%}:Denotes the percentage of reads to the L2 cache out of the total L2 accesses,
 {\it Write\%}:Denotes the percentage of writes to the L2 cache out of the total L2 accesses, {\it Intensity}: Read/Write intensive based on read\%/write\%.}
 \vspace{0.2in}
 \label{table:benchmark}
 \scriptsize
\begin{tabular}{||c|c|c|c|c||c|c|c|c|c||}
\hline
\# & PARSEC 2.1 & Read\% & Write\% & Intensity & \# & SPEC 2K6 & Read\% & Write\% & Intensity \\
\hline \hline
1 & blackscholes  &  91.9 &  8.1 &   Read &  13 &  bzip2 & 86.2 & 13.8 & Read \\
\hline
2 &     bodytrack  &     92.2 &  7.8 &   Read &  14 & gcc &  99.4 &  0.6 & Read \\
\hline
3 &     dedup  & 73.8 &  26.2 &  Write & 15 &    mcf & 94.5 &  5.5 & Read \\
\hline
4 &     facesim (fcsim.) & 78.7 &  21.3 &  Read &  16 & leslie3d & 70.7 & 29.3 & Write \\
\hline
5 &     ferret (frrt.) & 46.2 &  53.8 &  Write & 17 & namd &  92.7 &  7.3 & Read \\
\hline
6 &     fluidanimate (fluid.) &  82.4 &  17.6 &  Read &  18 & soplex & 59.6 & 40.4 & Write \\
\hline
7 &     freqmine (freq.) & 72.1 & 27.9 & Write & 19 & hmmer &   63.6 &  36.4 &  Write \\
\hline
8 &     rtview (rtvw.) & 64.1 & 35.9 &  Write & 20 & sjeng & 76.6 &  23.4 &  Write \\
\hline
9 &     streamcluster & 98.4 &  1.6 &   Read &  21 & libquantum(libq.) & 100.0 & 0.0 & Read \\
\hline
10 &    swaptions (swpts.) & 49.9 & 50.1 & Write & 22 & lbm & 15.7 & 84.3 & Write \\
\hline
11 &    vips &  75.0 &  25.0 &  Write &  23 & GemsFDTD & 99.2 & 0.8 & Read \\
\hline
12 &    x264 &  95.5 &  4.5 & Read &  24 & omnetpp & 97.7 & 2.3 & Read \\
\hline
& & & & & 25 &  h264ref & 57.8 &  42.2 &  Write \\
\hline 
\end{tabular}
\end{table}



\begin {table*} [t]
 \scriptsize
  \centering
 \caption {{Baseline processor, cache, memory and network configuration}} \label{table:sim_config}
 \begin{tabular}{|l|l|}
 \hline
Processor Pipeline & 2 GHz processor, 64-entry instruction window, Fetch/Exec/Commit width 8 \\
\hline
L1 Cache (SRAM) & 32 KB per-core (private) I/D cache, 4-way set associative, 64B block size, write-back, 10 MSHRs \\
\hline
L2 Cache (SRAM or STT-RAM) &  1MB (SRAM) or 4MB (STT-RAM) bank, shared, 16-way set associative, 64B block size, 10 MSHRs \\
\hline
Network & Ring network, one router per bank, 3 cycle router and 1 cycle link latency \\
\hline
Main Memory & 4GB DRAM, 400 cycle access \\
\hline
\end{tabular}
\end{table*}

%\noindent\textbf{Experimental Setup:}
We evaluate our designs using a modified ALPHA M5 Simulator \cite{M5} .
We operate the M5 Simulator in Full System (FS) mode for PARSEC 2.1 applications and in the System Emulation (SE) Mode
for the SPEC 2K6 multiprogrammed mixes. We model a 2GHz processor with four out-of order cores.
The memory instructions are modeled through M5 detailed memory hierarchy. We modified the M5 simulator to model
L2 cache banks composed of tunable retention time STT-RAM cells. Table~\ref{table:sim_config} details our experimental system configuration.

%\noindent\textbf{Collection of Results:}
We report results of 12 multithreaded PARSEC 2.1 applications and 14 multiprogrammed mixes of 4 SPEC 2K6 applications.
Multiprogrammed mixes are chosen randomly from the set of 13 SPEC 2K6 applications.
Table ~\ref{table:benchmark} shows the properties of PARSEC 2.1 and SPEC 2K6 applications.
We use sim-small input for PARSEC 2.1 applications and report the results of only Region of Interest,(ROI)(except for facesim, where we report results for only 2B instructions of ROI)
after warming up the caches for 500M instructions and skipping the initialization and termination phases. For the SPEC multiprogrammed mixes,
we fast forward 1B Instructions, warm up caches for 500M instructions and then report
results for 1B instructions.

\noindent\textbf{Design Choices:}
We report the results for the following L2 cache configurations:

%\squishlist

$\bullet$ {\bf S-1MB:} This is our baseline scheme, where all L2 cache banks are
composed of SRAM cells. Capacity of each bank is 1MB.

$\bullet$ {\bf S-4MB:} This is an ideal case, where all L2 cache banks
are composed of SRAM cells. Capacity of each bank is 4MB and each bank
has the same read and write latency as that of S-1MB. This case is analyzed to see the potential benefit
of having a 4x improvement in cache capacity at 4x area density, while still having read/write latencies comparable to SRAM.
This ideal but hypothetical design has the capacity, area and leakage properties of a STT-RAM but the read/write latencies of an SRAM cache.

$\bullet$ {\bf M-4MB:} This is our baseline scheme for STT-RAM design, where
all L2 cache banks are composed of 10 year retention time STT-RAM cells.
Capacity of each bank is 4MB and each bank occupies the same area as that of an SRAM bank.

$\bullet$ {\bf Volatile M-4MB(1sec):} This design is used to evaluate our
 Volatile STT-RAM Scheme described in Section~\ref{sec:implementation}, where all L2 cache
banks are composed of 1 sec retention time STT-RAM cells.

$\bullet$ {\bf Volatile M-4MB(10ms):} This design is similar to Volatile M-4MB(1sec)
except that the retention time of STT-RAM cells is 10 ms.

$\bullet$ {\bf Revived M-4MB(10ms):} This design is used to evaluate our
 Revived STT-RAM Scheme described in Section ~\ref{sec:implementation}, where all L2 cache
banks are composed of 10 ms retention time STT-RAM cells. All the
results are for the design with 8 MRU Slots and 1900 Buffer Slots.

All 4MB STT-RAM banks occupy the same area as that of a 1MB SRAM bank (because STT-RAM cells are 4x denser), but have varying write latencies due to varying retention times.
%\squishend

\noindent\textbf{ Evaluation Metrics:}
For the multithreaded PARSEC 2.1 applications, we assume 4 threads are mapped to our modeled
processor with four cores. We report normalized speedup for these applications,
which is defined as the improvement over the slowest thread.
For the multiprogrammed SPEC applications, we report Instruction Throughput and Weighted Speedup.
We define instruction throughput (IT) to be the sum of all the number of instructions committed
per cycle in the entire CMP (Eq. (5)).  The weighted speedup (WS) is defined as the slowdown experienced
by each application in a multiprogram mix, compared to its run under the same
configuration when no other application is running on the other cores(Eq.(6)). For analyzing the energy behavior, we measure the leakage energy, dynamic energy and total energy for all designs.

{
%\scriptsize{
 $Instruction$ $throughput$ $=$ $\displaystyle\sum_{i} IPC_{i}$ \hspace{1mm} \textbf{(5)} \hspace{4mm}
 $Weighted$ $speedup$ $=$ $\displaystyle\sum_{i}
\frac{IPC_{i}^{shared}}{IPC_{i}^{alone}}$ \hspace{1mm} \textbf{(6)}
}





%-------------------------------------------------------------------------
\section{Results} \label{sec:results}
In this section, we provide a comparative analysis of the performance and energy results of the six designs.
We also discuss the sensitivity of of several architectural parameters.
%experimental results associated with our design choices.



\begin{figure*} [t]
\centering
 \psfig{figure=figures/parsec-speedup.eps, width=6.9in, height=2.5in}
 \caption{\label{fig:parsec-new} \scriptsize \bf Normalized speedup for PARSEC Applications }
\end{figure*}

\begin{figure*} [t]
%\begin{figure*} [t]
\centering
 \psfig{figure=figures/writebacks.eps, width=4.9in, height=2.5in}
 \caption{\label{fig:writebacks} \scriptsize \bf Number of Write backs normalized to M-4MB}
%\end{figure*}
\end{figure*}


\subsection {Performance comparison}


Figure~\ref{fig:parsec-new} shows speedup results with multithreaded PARSEC
applications along with the average improvements. Only 9 applications are shown to reduce clutter in the plots, however,
the average numbers are computed across the entire suite. All speedup numbers are normalized to S-1MB.

{\bf Speedup when replacing a SRAM with an ideal STT-RAM cache}: When aggregated across the entire PARSEC suite, we find that this hypothetical design has an average speedup of 23\% over the SRAM cache. This is the maximum performance that any scheme can provide.

{\bf Speedup when replacing a SRAM with 4MB, 10 year retention time STT-RAM}: We find that, for the M-4MB, 10 year retention time STT-RAM design, all applications to the right of x264 (including x264) exhibit speedup improvements over S-1MB. Most of these applications are read intensive applications (details in Table~\ref{table:benchmark}) and thus, they not only benefit from the 4x capacity increase of STT-RAM, but also from the presence of a write buffer for the L2 cache. On an average, we find XX\% improvement in weighted-speed for these applications. Although ferret and vips are write-intensive applications, they benefit with a 4x improvement in capacity when going from a 1MB-SRAM to a 4MB-STT-RAM. This is because, the write request to L2 cache banks in these two applications are staggered in time. Thus, a 10-entry write buffer proves to be sufficient to hide the long write latencies of a STT-RAM bank.

All applications to the left of x264 are write-intensive and have have bursty requests arriving at L2 cache banks. For these applications, we observe significant degradation in speedup (on an average 11\% degradation ) because of the high write latency of STT-RAM. 
Overall, when averaged across the entire PARSEC suite, a traditional 10 year STT-RAM gives a minimal 5\% speedup improvement over S-1MB. However, a 10 year STT-RAM cache organization has 14\% lower performance when compared to the ideal design and with write-intensive applications this difference is 21\%. It is this gap that our proposal seeks to bridge by tuning the retention time

{\bf Speedup when replacing a SRAM with 4MB, 1 sec retention time STT-RAM}: With such a STT-RAM cache bank, no refreshing schemes are employed. As shown earlier, almost all blocks get refreshed within a 1 sec time interval. Reducing the retention time from 10 year to 1 sec reduced the write latency of a cache bank by 10 cycles (from 22 cycles to 12 cycles). This leads to significant speedup improvements over a 10 year retention time STT-RAM cache organization. On an average, this reduction in 10 cycles lead to 6\% performance improvement (14\% for write intensive applications). However, this design is still 9\% (11\% for write intensive applications) lower in performance than the ideal case.
Further, when compared to the baseline S-1MB SRAM design, most of the write intensive and bursty application have no speedup gain compared to
with this kind of STT-RAM cell.

{\bf Speedup when replacing a SRAM with 4MB, 10 ms retention time STT-RAM without refreshing}: This volatile M-4MB(10ms) design also does not have any refreshing scheme, but the retention time of STT-RAM cells used is 10ms. Using this STT-RAM device, after 10 ms the STT-RAM device will lose its data and hence to keep the data integrity on the CMP, use of this device forces a large number of write-backs from the last-level cache to the main-memory. Figure~\ref{fig:writebacks} shows number of write backs of all the designs normalized to M-4MB. We observe that the 4MB, 10 ms retention time STT-RAM design, on an average, has 21\% more write backs than the traditional STT-RAM design.
This leads to significant performance degradation across most applications when compared to simply using a 10 year retention time STT-RAM cache (8\% performance degradation on average). For instance, in vips, there is about 20\% speedup
degradation over M-4MB. It is interesting to see the case of swaptions, where there is a slight improvement in speedup over M-4MB, although
there is increase in number of write backs. The reason for this improvement is due to the fact that the majority of blocks that are not refreshed within 10ms interval, are not accessed in future as well leading to a low number of read misses. This helps in reaping benefits from the reduced write latency.

{\bf Speedup when replacing a SRAM with 4MB, 10 ms retention time STT-RAM with refreshing}: This is our proposed scheme (annotated as Rrevived-M-4MB(10ms) in the figures), which incorporates refreshing of dirty blocks beyond the 10ms retention time.
Use of this scheme, significantly improves performance when compared to all realistic design scenarios evaluated. On an average, the proposed revived scheme is better than the conventional SRAM design (S-1MB) by 18\%, traditional 10yr STT-RAM by 15\% and over volatile STT-RAM (1sec)
by 4.5\%. The write latency of this STT-RAM cache bank is 6 cycles and when compared to a 1 sec retention time STT-RAM, the difference of 6 cycles reduction in L2 cache bank access time significantly improves performance. This performance improvement is in spite of the increase in the number of write-backs (which increase by over 2x) compared to an SRAM cache. Fluidanimate is the only application we found that does not show a gain in performance with this type of STT-RAM device. With fluidanimate application, all the dirty blocks are almost equally distributed among all the ways, and hence our scheme of considering only the first eight MRU slots is insufficient in reducing many writebacks (as shown in Figure~\ref{fig:writebacks} with fluidanimate the writebacks become 11x with this volatile STT-RAM). 
%Facesim shows the maximum benefit when compared to the volatile STT-RAM (1 sec) cache: 22.7\%

The revived-M-4MB(10ms) scheme is closest to the ideal S-4M case and is within 4\% of it, showing the benefits of our scheme in making the STT-RAM device a choice for universal memory.

\begin{figure*} [t]
\centering
 \psfig{figure=figures/spec-ws.eps, width=3.9in, height=2.5in}
 \caption{\label{fig:spec-new} \scriptsize \bf Normalized Average Instruction Throughput(IT) and Weighted Speedup(WS) for SPEC 2006 multiprogrammed mixes. }
\end{figure*}

{\bf Analysis with SPEC 2006}: In general, the observations with SPEC benchmarks are consistent with those made with PARSEC applications. Figure~\ref{fig:spec-new} shows instruction throughput and weighted speedup with the  SPEC multiprogram
mixes. Simply replacing a SRAM bank with 4MB-STT-RAM would lead to 11\% degradation in instruction throughput, and 4\% degradation in weighted speedup. However, employing our refreshing scheme on a 10 ms volatile STT-RAM can lead to an average 22\%, 11\%  and 10\% improvement over M-4MB, volatile STT-RAM(1 sec) and baseline SRAM cache respectively. With a very write intensive mix (bzip2, gcc, lbm and hmmer) this improvement can be as high as 35\% over the baseline SRAM design. 

%Figure~\ref{fig:spec-new} shows instruction throughput and weighted speedup for the  SPEC multiprogram
%mixes. We observe that our scheme revived-M-4MB gives 22\% improvement in instruction throughput
%over M-4MB, 11\% improvement over Volatile STT-RAM (1 sec), and 10\% improvement over
%the base line SRAM cache. (although not shown in the Figure due to brevity, in case of the mix of
%bzip2, gcc, lbm, hmmer, the improvement is 15\%). The weighted speedup improvement over M-4MB is 4\%
%and over Volatile STT-RAM(1 sec) is 2\%. (NEED TO EXPLIAN WHY annd more insight).
%The weighted speedup improvement over M-4MB is 4\%
%and over Volatile STT-RAM(1 sec) is 2\%. (NEED TO EXPLIAN WHY annd more insight).

Weighted speedup also follows a similar trend as instruction throughput and we find that our proposed refreshing scheme on a 10 ms retention time volatile STT-RAM shows the best results. Overall, our proposed design in within 9\% (2\%) of the ideal device in terms of instruction throughput(weighted speedup). 

\subsection{Energy comparison}

\begin{figure*} [t]
\centering
\begin{tabular}{c}
\psfig{figure=figures/legend.eps, width=5.5in, height=0.15in}
\end{tabular}
\begin{tabular}{ccc}
 \psfig{figure=figures/leak-eng.eps, width=2.1in, height=2.0in} &
\psfig{figure=figures/dyn-eng.eps, width=2.1in, height=2.0in} &
\psfig{figure=figures/tot-eng.eps, width=2.1in, height=2.0in} \\
\scriptsize (a) Leakage Energy  & \scriptsize (b) Dynamic Energy & \scriptsize (c) Total Energy
\end{tabular}
 \caption{\scriptsize \bf Energy of Applications Normalized to that of S-1MB}
\label{fig:energy}
\end{figure*}

Figure~\ref{fig:energy} shows normalized cache leakage, total of dynamic read and write energy, and total energy for a subset of PARSEC applications. While computing the energy numbers, we take into account the overheads of our proposed cache block refreshing (revived) schemes.
%The number of reads and writes to L2 cache are only considered for the calculation of dynamic energy.
We observe that on an average there is 44\% improvement in total energy going from S-1MB to
M-4MB designs. This improvement is mainly because of the drastic reduction in leakage energy (XX\%).
In general all volatile STT-RAMs reduce the energy envelope of the last level cache. With 1 sec volatile STT-RAM, 
total energy is reduced because of reduction in leakage energy and also nominal performance improvement. However, when comparing
volatile M-4MB(10ms) with volatile M-4MB(1s), because of larger number of write-backs with 10 ms retention time STT-RAMs, the performance degrades and thus, leakage energy increases.

With our proposed cache block refreshment scheme, we consistently observe improvement in total energy (due to both reduction in leakage power and improvement in performance). on an average, we find 11\% energy benefits of using revived M-4MB design over Volatile-1sec and 30\% improvement over baseline STT-RAM design. 

In conclusion, we find that our proposed revived scheme is significantly better in terms of performance and energy when compared to a traditional non-volatile STT-RAM and even a volatile STT-RAM with 1 sec retention time. Further, the proposed 10 ms retention time STT-RAM with revival schemes, is very close to the ideal last-level cache architecture. This makes our proposed STT-RAM device an attractive contender for the universal on-chip memory.


%significantly improves performance of STT-RAM banks, and bridges the gap between traditional STT-RAMs and the maximum theoretical limit. We observe that our scheme is better in terms of both performance and energy over Volatile-4MB(1sec) and
%M-4MB designs.

%Volatile M-4MB(1sec) leakage benefits over M-4MB correlates with the performance improvement.
%On an average, this design consume more dynamic energy than M-4MB. The dynamic energy fluctuations among
%different applications are on account of changes in number of read and writes. Additional write backs
%triggers read misses which ultimately lead to additional writes to L2 cache. Write energies of 1sec design is more
%than the 10ms designs, which makes the fluctuations depend on the number of reads/writes.
%We see 11\% energy benefits of using revived M-4MB design over Volatile-1sec and 30\% improvement over
%M-4MB designs.The energy numbers of this scheme covers all the overheads of  the buffer design.
%We observe that our scheme is better in terms of both performance and energy over Volatile-4MB(1sec) and
%M-4MB designs.




\subsection{Sensitivity Analysis}
In this subsection, we study the sensitivity of various architectural parameters we chose in proposing the cache revival architecture.

\noindent\textbf{Sensitivity to number of buffer entries:} 
The number of buffer entries can affect the performance of Revived-M-4MB scheme in two ways: increasing the 
buffer size will accommodate more diminishing blocks at a particular instance and decreasing the buffer size will
leads to more buffer overflows. With increasing the buffer size, leading to fewer buffer overflows, 
the reduction comes at a cost of increase in buffer area and consequent revival overheads. Decreasing the buffer size, eventually
leads to additional write backs (discussed in~\ref{sec:implementation}).

\begin{figure*} [t]
\centering
 \psfig{figure=figures/confi.eps, width=3.9in, height=2.5in}
 \caption{\label{fig:confi} \scriptsize \bf 95\% Confidence Intervals of Diminished Blocks for each Way}
\end{figure*}

To find the optimal buffer size, we calculate the 95\% confidence intervals (CIs) for the cumulative distribution of
dead blocks per bank. This is shown in~\ref{fig:confi}. We observe that, for first 8 MRU slots,
mean value of the buffer entries is 1900 blocks, which corresponds to a 3\% area overhead per L2 cache bank.
Upper limit of the 95\% CI corresponds to 2500 blocks,  which represents 4\% area overhead per L2 cache bank. 
The lower limit of 95\% CI corresponds to 1300 blocks (2\% area overhead). 

Figure~\ref{fig:buf-mru} shows the speedup with a subset of PARSEC applications by varying the number
of buffer entries. Going from 1900 to 2500 entries results in only less than 0.5\% speedup improvement. 
Hence, in all our results, we used 1900 buffer entries (resulting in a 3\% area overhead) resulting in the best possible performance per area-overhead.

\begin{figure*} [t]
\centering
\begin{tabular}{cc}
 \psfig{figure=figures/buffer.eps, width=3.4in, height=2.0in} &
\psfig{figure=figures/slots.eps, width=3.4in, height=2.0in} \\
\scriptsize (a) Buffer Entries & \scriptsize (b) MRU Slots
\end{tabular}
 \caption{\scriptsize \bf Showing effects on speedup by varying number of Buffer Entries and MRU Slots }
\label{fig:buf-mru}
\end{figure*}


\noindent\textbf{Sensitivity to number of MRU slots:}
Choosing optimal number of MRU slots to buffer was discussed briefly in~\ref{sec:implementation}. 
Figure~\ref{sec:cdf} shows that mean cumulative distribution of diminishing blocks per 
way across all PARSEC benchmarks. We see that after 8 MRU slots, the number of diminishing blocks 
becomes negligible, which suggests that optimal number of MRU slots is 8. Figure~\ref{fig:buf-mru} shows
speedup of subset of PARSEC applications, along with the average across 12 PARSEC applications, with varying
number of MRU slots. Buffer size is kept constant at 1900 per bank. We see degradation in performance when we decrease
slots from 8 to 4 since, buffering 8 MRU slots would have covered more frequently used blocks and hence, reducing
write backs of useful blocks. We also see degradation in performance by increasing slots from 8 to 12. This is because, with coonstant buffer
size, 12 MRU slots increase the probability of buffer overflows, which increases the write backs leading to performance degradation. 

\noindent\textbf{Sensitivity to number of bits of the counter:}
As discussed in~\ref{sec:implementation} increasing the number of bits of the counter decreases the
left over time at the cost of incrementing counters at finer granularity. Our experiments showed that
there is no observable difference in performance and energy by increasing/decreasing the number of bits of the counter.









%-------------------------------------------------------------------------
\section{Prior Work} \label{sec:prior_work}
%[ADWAIT, MORE STT-RAM RELATED WORK other than HPCA 2011 paper ESPECIALLY FOR STT-RAM CACHE DESIGN should be put here]

%Liang et al.~\cite{3t1d-cache} studied the impact of process variation on a 3T1D
%L1 data cache and modeled the process variations as variations in data retention time.

%Below, we discuss the prior works most relevant to our work.

%This section summarizes the prior work closely related to ours.
This section summarizes the circuit and architectural techniques proposed for enhancing the STT-RAM
write performance.

The work that is most closely related to ours is~\cite{STTRAM:HPCA11}.
Here, the authors relax the retention time of STT-RAM from $10 years$ to $56 \mu s$ by
reducing the planar area of MTJ from $32F^2$ to $10F^2$. However this methodology is limited by 
technology challenges in newer scaled MTJs. Recently published work related to
STT-RAM~\cite{PMTJ:Toshiba08,STTRAM:EDL11,STTRAM:Qualcomm09,STTRAM:Grandis11}
uses state-of-the-art MTJ designs in the range of 2-3$F^2$. Since, these designs don't give the
leeway of reducing the retention time by aggressively reducing the MTJ planar area as in~\cite{STTRAM:HPCA11} , we reduce the retention time by lowering the saturation magnetization and the thickness of the free layer. Analysis of retention times of LLC blocks with 25 multi-threaded and multi-programmed applications,
as compared to only 5 applications in~\cite{STTRAM:HPCA11}, shows that
the ideal retention time of LLC should be in the order of $ms$ and not in {\it $\mu$s} as proposed
in their work. Moreover, the thickness of the free layer for the MTJs in our work is 
around $2nm$, and reducing the thickness further
to get {\it $\mu$s} retention time, will lead into MTJ reliability issues
and make it more vulnerable to process variations. Also, our proposed refresh scheme is simple, yet very performance and energy efficient compared to the DRAM-style refreshing proposed in~\cite{STTRAM:HPCA11}.

The 3T1D designs proposed in ~\cite{3T-brooks} has typical worst-case retention time in {\it $\mu$s} region, which makes it incompatible for our LLC design that needs $ms$ retention time. Increasing the retention time of 3T1D cell to $ms$ region will enlarge the size of the gated-diode or will increase the threshold voltage of the access transistor. These choices will incur significant area overhead or performance degradation for our cache design, respectively. 

Before submitting this paper, we came across a new work that explores the possibility of designing LLC with STT-RAM banks of varying retention times and moving dying blocks from lowest retention time ({\it $\mu$s}) bank to the higher ones~\cite{multi-level-retention}. As described before, fabricating STT-RAM cells with {\it $\mu$s} retention time has the same fabrication and reliability issues. Moreover, including different types of STT-RAM cells corresponding to different retention times in the same hierarchy will impose greater challenge and incur higher die cost. Also, to moderate our results we considered the fact that all cache banks may leak and reported leakage numbers asssuming no power gating techniques have been applied.

Apart from this recent work, a few other prior works have also proposed architectural and circuit level
solutions to handle this long write latency problem in STT-RAMs. Architectural techniques such as
early write termination~\cite{mram-energy-reduction}, hybrid SRAM/STT-RAM
architecture~\cite{gsun-hpca, Qureshi:2009:SHPMM} and read-preemptive write-buffer designs have been
shown to mitigate write latency/energy. The circuit level techniques such as eliminating redundant
bit-writes~\cite{mram-energy-reduction} and data inverting technique~\cite{gsun-hpca} have also been
shown to be effective in hiding the long write latency. In contrast to all these prior works that
attempt to {\it hide} the write latency, our scheme investigates techniques to {\it actually} reduce
the write latency of STT-RAM banks and make their write latency comparable to SRAM banks. When
compared to Zhou et al.'s work~\cite{mram-energy-reduction} that require additional gates for
detection and termination of writes inside {\it each STT-RAM sub-bank}, our techniques are simpler to
implement since our proposal works at a much coarser granularity.

Sun et al.~\cite{gsun-hpca} showed that write buffers can be helpful in hiding the long write
latencies of STT-RAM banks. Our analysis shows that, if an application is bursty, write-buffers fail
to hide this latency and are rendered in-effective. Out of 25 applications, we found 12 applications
to be write intensive and bursty and hence, write-buffering is ineffective for these applications.
Moreover, all our results are conservative since we have already assumed a 10-entry (as used
in~\cite{gsun-hpca}) write-buffer at every STT-RAM bank and our results would be significantly better
without the presence of write-buffers.

In a recent work~\cite{mram-noc}, the authors have proposed a network level solution to hide the
write latency of STT-RAM banks. This solution requires complex busy/idle bank detection followed by
prioritization mechanisms in the network. On a qualitative basis, the network level solution to hide
write latency in \cite{mram-noc} was shown as the most promising technique compared to any other
techniques. The application level performance improvement with this scheme was about 2-4\% higher
compared to the write buffering technique of Sun et al.~\cite{gsun-hpca}. Contrasting this to our
work here, our scheme provides about 15\%/4\%(PARSEC IPC/SPEC weighted-speedup) improvement over $10years$
traditional STT-RAM, on top of the write buffering scheme, thereby making it more attractive compared
to \cite{mram-noc}. Overall, we believe that no prior work makes a case for tuning the retention time
of STT-RAM banks that is based on profiling retention duration of last-level cache blocks of
applications, which our proposal does.

%-------------------------------------------------------------------------
\section{Conclusions} \label{sec:conclusions}
Spin-Transfer Torque RAM (STT-RAM) is a promising candidate for future
on-chip cache design due to its high-density, low leakage, and immunity to
soft errors.  However, it's high write latency and dynamic
write energy are the disadvantages compared to SRAM-based cache design.
In this paper, we propose to trade-off the non-volatility (data retention time)
for better write performance/energy in STT-RAM cache design.
In this context, we conduct an application-driven study to characterize the
life time of a second level cache with the intention of using this time as the ideal
retention time for the STT-RAM. Execution-driven experiments with several PARSEC and SPEC benchmarks
indicate that at leat 50\% of the cache blocks are updated in 10ms and thus, chose 10 ms
as an optimal retention time by analyzing the STT-RAM retention time and write time trade-offs.
We investigate two design alternatives for avoiding the data loss due to the volatile nature of the
STT-RAM. The first approach write backs all the dirty blocks in the cache at the end of the retention time
and the second approach uses a buffering scheme to refresh the cache blocks that are not refreshed
during the retention time.

We analyze three different scenarios for designing the L2 cache: one with 1 second retention time with write back,
second with 10ms retention time with write back and the third with 10ms retention time with buffering, called
revived-STT-RAM. Compared to a base case design of 1MB per core SRAM design, the traditional non-volatile STT-RAM
cache with 4 times the SRAM capacity but high write latency, and the volatile STT-RAM with simple write back policy,
the proposed revive scheme shows both performance and power benefits across the application benchmarks studied in
this paper. The results not only indicate that it is possible to get up to XX\% improvement in instruction
throughput and YY\% reduction in total energy consumption, the proposed design can be within 4\% of the ideal case
with an equal capacity SRAM configuration, while being more energy efficient.
Furthermore, compared to the prior schemes that are aimed at hiding the high write latency of STT-RAMs,
the approach to reduce its write latency seems a better solution for designing a performance and power efficient
memory hierarchy for multi-cores. 

NOTE: ADD FINAL CONCLUSION

%-------------------------------------------------------------------------


%%%%%%%%%%%%%%%%%%%%%%%%%%%%%%%%%%%
%ASIT: \etal.

%\begin{quotation}
%\noindent
%   It displayed the following behaviours
%   which show how well we solved cases A and B: ...
%\end{quotation}

%\noindent

%%%%%%%%%%%%%%%%%%%%%%%%%%%%%%%%%%%%%%%%%%%%%%%%%%
{ 
\singlespacing
{\small
\bibliographystyle{ieee}
\bibliography{bibtex/mram,bibtex/cacti,bibtex/sttram,bibtex/others,bibtex/architecture,bibtex/MICRO}
}
}


\end{document}
