\singlespacing

Spin-Transfer Torque RAM (STT-RAM) is an emerging non-volatile memory (NVM)
technology that has the potential to replace the conventional
on-chip SRAM caches for designing a more efficient memory hierarchy for
future multi-core architectures.
%While the high density, low leakage and high endurance
%of STT-RAM are attractive compared to SRAM,
However, it's long write latency and high dynamic
write energy are major obstacles for being competitive with the SRAM-based cache hierarchy.
On the other hand, the non-volatility feature with years of data retention time for STT-RAM technology is not necessary for the usage of STT-RAM as on-chip cache, since the life time of cache data are usually within us or ms. Consequently,  we exploit such observation for designing an efficient L2 cache
architecture, and propose to trade off the non-volatility (data retention time)
for better write performance/energy in STT-RAM cache design. The paper addresses several critical design issues such as how do we decide a suitable retention time for last level cache,
what is the relationship between retention time and write latency,
and how do we architect the cache hierarchy with a volatile STT-RAM.
%Through an extensive execution driven analysis of the inter-write time of several PARSEC and SPEC 2006 benchmarks, we
%observe that retention time in the order of 10-40 ms is a good design point to handle most of the
%writes.
We study two
data-retention relaxation cases, one with data retention time of 1 second, which satisfies the lifetime requirement of typical cache blocks; and the other one with data retention time of 1ms, which is a more aggressive design for better performance/energy gains but a data refreshing mechanism is needed.
In the aggressive data retention time relaxation design, for the rest of the cache blocks that have a higher inter-write
time than the STT-RAM retention time, we propose an architectural solution to identify these blocks
with a per block 2 bit counter, temporarily save a limited number of MRU blocks in a buffer,
and write-back the rest of the dirty blocks to avoid any data loss.
%are not in first seven MRU slots.
%The blocks in these slots are copied to
%a per bank small buffer and again copied back to the respective slots.
Our experiments with 4 and 8-core architectures with an SRAM-based L1 cache and STT-RAM-based L2 cache
indicate that not only we can eliminate the high write overhead of an NVM STT-RAM, but can provide
on an average 10-12\% improvement in IPC compared to the traditional SRAM-based
design, while reducing the energy consumption significantly

