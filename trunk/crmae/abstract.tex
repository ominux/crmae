
Spin-Transfer Torque RAM (STT-RAM) is a CMOS compatible emerging non-volatile memory (NVM) 
technology that has the potential to replace the conventional
on-chip SRAM caches for designing a more efficient memory hierarchy for
future multicore architectures. 
%While the high density, low leakage and high endurance
%of STT-RAM are attractive compared to SRAM, 
However, it's high write latency and dynamic
write energy are major obstacles for being competitive with the SRAM-based cache hierarchy.
On the other hand, STT-RAM technology has another adaptable feature that it is possible to reduce its write
latency by reducing its retention time, thereby making it volatile.
In this paper, we exploit this volatile property of the STT-RAM for designing an efficient L2 cache 
architecture. The paper addresses several critical design issues such as how do we decide a suitable retention time for L2 cache,
what is the relationship between retention time and write latency,
and how do we architect the cache hierarchy with a volatile STT-RAM.
Through an extensive execution driven analysis of the inter-write time of several PARSEC and SPEC banchmarks, we 
observe that retention time in the order of 10s of ms is a good design point to handle most of the
writes.  Then for the rest of the cache blocks that have a higher inter-write
time than the STT-RAM retention time, we propose an architectural solution to identify these blocks
with a per block 2 bit counter, temporarily save a limited number of MRU blocks in a buffer,
and invalidate the rest of the dirty blocks to avoid any data loss.
%are not in first seven MRU slots.
%The blocks in these slots are copied to
%a per bank small buffer and again copied back to the respective slots.
Our experiments with 4 and 8-core architectures with an SRAM-based L1 cache and STT-RAM-based L2 cache 
indicate that not only we can eliminate the high write overhead of an NVM STT-RAM, but can provide
on an average 10-12\% improvement in IPC compared to the traditional SRAM-based
design, while reducing the energy consumption siginficantly. 

