

\begin {table*} [t]
 \scriptsize
  \centering
 \caption {{\bf Baseline processor, cache, memory and configuration}} \label{table:sim_config}
 \begin{tabular}{|l|l|}
 \hline
Processor Pipeline & 2 GHz processor, 64-entry instruction window, Fetch/Exec/Commit width 8 \\
\hline
L1 Caches & 64 KB per-core (private), 4-way set associative, 64B block size, write-back, split I/D caches, 10 MSHRs \\
\hline
L2 Caches &  1MB banks, shared, 16-way set associative, 64B block size, 4-cycle bank latency, 10 MSHRs \\
\hline
Main Memory & 4GB DRAM, up to 16 outstanding requests for each processor, 400 cycle access \\
\hline
\end{tabular}
\end{table*}

\noindent\textbf{Experimental Setup}
We evaluate our design and schemes on the modified  ALPHA M5 Simulator \cite{} . We operate M5 Simulator in Full System (FS) mode for PARSEC applications and in System Emulation (SE) Mode for SPEC 2006 Multiprogammed mixes. We model a 2GHz processor with four out of order cores. We modified M5  simulator to model low retention time STT-RAM for L2 cache. The L2 cache  is banked with different read and write latencies, with all the banks connected via a shared memory bus. We assume a fixed 400 cycles main memory latency for all our simulations. Table~\ref{table:sim_config} details our experimental system configuration. 

\noindent\textbf{Collection of Results}
We report results of 12 multithreaded PARSEC applications and 14 SPEC 2006  multiprogrammed mixes. Table ~\ref{fig:benchmarks} shows the characterization of PARSEC applications and list of multiprogrammed mixes.
We use sim-small input for PARSEC benchmarks and report the results of only Region of Interest (ROI) after skipping the initialization and termination phases (except facesim, where we report results for only 2B instructions of ROI) We also warm up caches for 100M Instructions in ROI. For SPEC multiprogrammed mixes, we fast forward 1B Instructions, warm up caches for 500M instructions and then report results for 1B instructions. 

\noindent\textbf{Performance Metrics}
For multithreaded PARSEC applications, we assume 4 threads are mapped to our modeled processor with four cores. We report normalized speedup for these applications, which is defined as the decrease in execution time of the slowest thread. We randomly choose 14 multiprogram mixes, each mix with four different applications and assign each every application to a core. 
We report Instruction throughput and Weighted Speedup for SPEC multiprogrammed mixes, which are defined as:

