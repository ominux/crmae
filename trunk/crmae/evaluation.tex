

\begin {table*} [t]
 \scriptsize
  \centering
 \caption {{\bf Baseline processor, cache, memory and network configuration}} \label{table:sim_config}
 \begin{tabular}{|l|l|}
 \hline
Processor Pipeline & 2 GHz processor, 64-entry instruction window, Fetch/Exec/Commit width 8 \\
\hline
L1 Cache (SRAM) & 32 KB per-core (private) I/D cache, 4-way set associative, 64B block size, write-back, 10 MSHRs \\
\hline
L2 Cache (SRAM or STT-RAM) &  1MB (SRAM) or 4MB (STT-RAM) bank, shared, 16-way set associative, 64B block size, 10 MSHRs \\
\hline
Network & Ring network, one router per bank, 3 cycle router and 1 cycle link latency \\
\hline
Main Memory & 4GB DRAM, 400 cycle access \\
\hline
\end{tabular}
\end{table*}

%\noindent\textbf{Experimental Setup:}
We evaluate our designs using a modified ALPHA M5 Simulator \cite{M5} . 
We operate the M5 Simulator in Full System (FS) mode for PARSEC applications and in the System Emulation (SE) Mode
for the SPEC 2K6 Multiprogammed mixes. We model a 2GHz processor with four out of order cores.
The memory instructions are modeled through M5 detailed memory hierarchy. We modified the M5 simulator to model
L2 cache banks composed of tunable retention time STT-RAM cells. A fixed 400 cycles main memory latency is used for all our simulations. Table~\ref{table:sim_config} details our experimental system configuration. 

%\noindent\textbf{Collection of Results:}
We report results of 12 multithreaded PARSEC applications and 14 multiprogrammed mixes of 4 SPEC 2K6 applications. 
Multiprogrammed mixes are chosen randomly from the set of 13 SPEC 2K6 applications.
Table ~\ref{fig:benchmarks} shows the properties of PARSEC and SPEC 2K6 applications.
We use sim-small input for PARSEC applications and report the results of only Region of Interest (ROI) 
after warming up the caches for 500M instructions and skipping the initialization and termination phases 
(except facesim, where we report results for only 2B instructions of ROI). For the SPEC multiprogrammed mixes,
we fast forward 1B Instructions, warm up caches for 500M instructions and then report
results for 1B instructions. 

\noindent\textbf{Design Choices:}
We report the results for the following designs:
\squishlist
\item {\bf S-1MB:} This is our baseline scheme, where all L2 cache banks are
composed of SRAM cells. Capacity of each bank  is 1MB. 
\item {\bf S-4MB:} This is our ideal case, where all L2 cache banks
are composed of SRAM cells. Capacity of each bank is 4MB and each bank
has same read and write latency as that of S-1MB.
\item {\bf M-4MB:} This is our baseline scheme for STT-RAM design, where
all L2 cache banks are composed of 10 year retention time STT-RAM cells. 
Capacity of each bank is 4MB.  
\item{\bf Volatile M-4MB(1sec):} This design is used to evaluate our 
 Volatile STT-RAM Scheme described in ~\ref{}, where all L2 cache 
banks are composed of 1 sec retention time STT-RAM cells. 
\item {\bf Volatile M-4MB(10ms):} This design is similar to Volatile M-4MB(1sec)
except that, now the retention time of STT-RAM cells is 10 ms.
\item {\bf Revived M-4MB(10ms):} This design is used to evaluate our 
 Revived STT-RAM Scheme described in ~\ref{}, where all L2 cache 
banks are composed of 10 ms retention time STT-RAM cells. All the
results shown with design uses 8 MRU Slots and 2048 Buffer Slots. 
\squishend

\noindent\textbf{Performance Metrics:}
For the multithreaded PARSEC applications, we assume 4 threads are mapped to our modeled
processor with four cores. We report normalized speedup for these applications, 
which is defined as the improvement over the slowest thread. 
For the multiprogrammed SPEC applications, we report Instruction throughput and Weighted Speedup.
We define instruction throughput (IT) to be sum total of all the number of instructions committed
per cycle in the entire CMP (Eq. (5)).  The weighted speedup (WS) is defined as the slowdown experienced
by each application in a multiprogram mix, compared to its run under the same
configuration when no other application is running on the other cores.(Eq.(6)).

{
%\scriptsize{
 $Instruction$ $throughput$ $=$ $\displaystyle\sum_{i} IPC_{i}$ \hspace{1mm} \textbf{(5)}

 $Weighted$ $speedup$ $=$ $\displaystyle\sum_{i}
\frac{IPC_{i}^{shared}}{IPC_{i}^{alone}}$ \hspace{1mm} \textbf{(6)}

}

