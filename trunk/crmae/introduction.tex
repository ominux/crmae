
Spin-Transfer Torque RAM (STT-RAM) is a promising memory technology that
delivers on many aspects desirable of an universal memory. They exhibit
high density, fast read times and low static power consumption. However,
the high write latencies and write energy are key drawbacks of this
technology. Consequently, recent efforts have focused on masking the
effects of high write latencies and write energy at the architectural level.
In contrast to these architectural approaches, a recent technique
considers relaxing STT-RAM data retention times to reduce both write
latencies and
write energy. The focus of this paper is  to tune this data retention
time to closely match the required lifetime of cache line blocks to
achieve significant
performance and energy gains.

The non-volatile nature and non-destructive read ability of  STT-RAM
provides a key difference with regard to a comparably high-density DRAM
memory. However, for many applications it is sufficient if the data
stored in a cache hierarchy remains valid for a few tens of
milliseconds. Consequently, the duration of data retention in STT-RAM is
an obvious candidate for device optimization for cache design. First,
we  analyze how changes to the STT-RAM retention times influence the
performance, power and area characteristics. A key distinction from
prior efforts to relax data retention times is our consideration of
device variability in our analysis. 

Analyzing the lifetime of cache lines has been the focus of prior
efforts to improve performance and reduce power consumption. In this
work, we
revisit this topic with the aim of identifying the suitable data
retention times for STT-RAM caches. A key challenge in determining a
suitable
data retention times for the STT-RAM is to balance the reduced write
latency of cells with lower retention time with the overhead for data
refresh or
writeback of cache lines with longer lifetimes. Our analysis of the
cache lifetimes performed for a multi-threaded workload demonstrates
that a significant fraction of L2 cache lines can operate correctly
without any additional support when the STT-RAM retention times are of
the order of 50ms. However, architectural support is required to ensure
that correct program state is maintained for the rest of the cache lines
that have lifetimes exceeding 50ms. While a simple
DRAM-style refresh has been proposed in~\cite{STTRAM:HPCA11} to ensure
correctness, it is possible to avoid many of these refresh by pursuing a
life-time aware refresh strategy.

This work makes the following contributions

 - We present a detailed device characterization of data retention
tunability in STT-RAM Cells providing insight to the underlying principles
enabling these tradeoffs

- We analyze the time between writes or replacements to a cache line for
various multi-threaded and multi-programmed workloads. Our characterization
augments the prior body of work that analyzes cache lifetimes mainly in
single processor and single program configurations.

- We present a simple buffering mechanism to ensure integrity of
programs given the volatile nature of our tuned STT-RAM cells.

- Finally, we show that our combined device-architecture life time
tuning approach is better than recent efforts that attempt to address the
long write latencies of STT-RAM.

The rest of this paper is as follows.






Next, we evaluate the lifetimes of cache lines when executing
multi-threaded workloads. A cache line is
no longer required, if the


for the duration of its lifetime in the program execution.

The cache hierarchy is a key component influencing both the 