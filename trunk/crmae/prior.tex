[ADWAIT, MORE STT-RAM RELATED WORK other than HPCA 2011 paper ESPECIALLY FOR STT-RAM CACHE DESIGN should be put here]

Moreover, in~\cite{STTRAM:HPCA11} the authors relax retention time of STT-RAM from $10 years$ to $56 \mu s$ by reducing the planar area of MTJ from $32F^2$ to $10F^2$. However, the scope of this work is limited by addressing practical device parameters and their variabilities. First, the retention time of MTJ is exponentially proportional to the thermal barrier, which makes the retention time of individual STT-RAM device extremely sensitive to any factor that has impact on thermal barrier, particularly device geometry. Thus it's important to take practical values of device geometry such as MTJ planar area and take their process variations into consideration. We get these parameters and corresponding variabilities from fabricated STT-RAM published in recent years~\cite{PMTJ:Toshiba08,STTRAM:EDL11,STTRAM:Qualcomm09,STTRAM:Grandis11}. These state-of-the-art MTJs has much smaller baseline planar area that is around $2F^2$. Therefore there is not too much room to reduce retention time by aggressively reduce MTJ planar area. In this paper, we focus on the MTJ with worst-case retention time larger than millisecond and optimize STT-RAM cache correspondingly. Our analysis in this paper reveals the granularities at which a device designer can reliably tune the data retention times. 