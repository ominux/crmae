
Designing an efficient memory hierarchy for multicore architectures is a critical but challenging
problem. As the number of cores on a chip increases with technology scaling, the demand on the
on-chip memory would increase significantly, thereby worsening the memory wall problem
~\cite{BurgerGK96}. The memory wall problem is critical both from the performance (memory density)
and power perspectives. Thus novel technology, circuit and architectural techniques are currently
being explored to address the memory wall problem for many core systems.

Spin-Transfer Torque RAM (STT-RAM) is a promising memory technology that delivers on many aspects
desirable of an universal memory. It has the potential to replace the conventional on-chip SRAM
caches because of its higher density, competitive read times and lower leakage power consumption
compared to SRAM. However, the high write latencies and write energy are key drawbacks of this
technology for providing competitive or better performance compared to the SRAM-based cache
hierarchy. Consequently, recent efforts have focused on masking the effects of high write latencies
and write energy at the architectural level \cite{}. In contrast to these architectural approaches, a
recent  work explored the feasibility of relaxing STT-RAM data retention times to reduce both write
latencies and write energy \cite{}. This adaptable feature of tuning the data retention time can be
exploited in several dimensions. The focus of this paper is  to tune this data retention time to
closely match the required lifetime of the last level cache line blocks to achieve significant
performance and energy gains. In this context, the paper addresses several design issues such as how
to decide an appropriate retention time for the last level cache, what is the relationship between
retention time and write latency, and how do we architect the cache hierarchy with a volatile
STT-RAM.

The non-volatile nature and non-destructive read ability of  STT-RAM provides a key difference with
regard to traditional on-chip cache design with SRAM technology. However, for many applications, it
is sufficient if the data stored in the last level of a cache hierarchy remains valid for a few tens
of milliseconds. Consequently, the duration of data retention in STT-RAM is an obvious candidate for
device optimization for cache design. We, therefore, conduct an application-driven study to analyze
the inter-write times of the L2 cache blocks to decide a suitable data retention time. Although
lifetime analysis of cache lines has been the conducted earlier to improve performance and reduce
power consumption \cite{}, we revisit this topic with a different intention - correlating STT-RAM
data retention time to cache life time. An extensive analysis of PARSEC and SPEC benchmarks using the
M5 simulator \cite{} indicates that the average inter-write times for most of the L2 cache blocks is
around 10ms, and thus, we advocate our STT-RAM design with this retention time.

We conduct a  detailed device level analysis of the STT-RAM cells to analyze the write current
versus write pulse width tradeoffs, cell area analysis, and retention time stability analysis
to capture the relationship between area, read/write latency and leakage power as a function of
the retention time. Our observations, in contrast to the results reported in \cite{} indicate
that retention times in the range of ms are probably more achievable than in the micro second range.
Thus, using this ms range retention time model, we then propose effective architectural techniques to
avoid any data loss due the volatile STT-RAM-based cache hierarchy.

A key challenge in determining a suitable data retention times for the STT-RAM is to balance the
reduced write latency of cells with lower retention time with the overhead for data refresh or write
back of cache lines with longer lifetimes. In this paper, we compare 3 different STT-RAM based cache
designs: (1) STT-RAM cache without retention time relaxation (>10 years of data retention time); (2)
STT-RAM cache with retention time of 1 second, which is long enough for the lifetime of majority of
the cache lines and therefore no refreshing overhead is incurred; (3)STT-RAM cache with retention
time of 10ms, which is a more aggressive design with better performance/energy gain but a data
refreshing technique is needed for correct operations since cache lines that have lifetimes exceeding
10ms are likely to loose data. Thus, we propose simple extensions to the L2 cache design for avoiding
any data loss. This include a simple 2-bit counter like the victim cache \cite{} to keep track of the
lifetime of all the cache blocks and a small buffer to temporarily store the blocks whose time has
exceeded the retention time.
%cache lifetimes performed for a multi-threaded workload demonstrates
%that a significant fraction of L2 cache lines can operate correctly
%without any additional support when the STT-RAM retention times are of
%the order of 50ms. However, architectural support is required to ensure
%that correct program state is maintained for the rest of the cache lines
%that have lifetimes exceeding 50ms. While a simple
%DRAM-style refresh has been proposed in~\cite{STTRAM:HPCA11} to ensure
%correctness, it is possible to avoid many of these refresh by pursuing a
%life-time aware refresh strategy.
We conduct execution-driven analysis of our proposed techniques using the M5 simulator and a suite of
PARSEC and SPEC benchmarks. The main contributions of this work are the following:

\noindent\textbf{Detailed characterization of STT-RAM volatile property:} We present a detailed
device characterization of data retention tunability in STT-RAM Cells providing insight to the
underlying principles enabling these tradeoffs. We believe the design in \cite{ } is very aggressive
and may not be feasible considering the state-of-the-art in device technology. On the other hand, if
the micro second level retention time is feasible, then our proposed architectural solution would be
even more beneficial.

\noindent\textbf{An application-driven study to determine retention time:}
We analyze the time between writes or replacements to a cache line for
various multi-threaded and multi-programmed workloads. Our characterization
augments the prior body of work that analyzes cache lifetimes mainly in
single processor and single program configurations. Based on the L2 cache
behavior, we propose to design STT-RAMs with retention time in the range of 10ms.

\noindent\textbf{Architectural solution to handle STT-RAM volatility:} We present a simple buffering
mechanism to ensure the integrity of programs given the volatile nature of our tuned STT-RAM cells.
Experimental results with PARSEC and SPEC benchmarks on a four-core and eight-core multicore platform
compared to a base case 1MB SRAM per core and the ideal 4MB SRAM per core indicate that the proposed
solution is attractive both from performance and power perspectives. (Summarize the results here)

%- Finally, we show that our combined device-architecture life time
%tuning approach is better than recent efforts that attempt to address the
%long write latencies of STT-RAM.

The rest of this paper is as follows:
