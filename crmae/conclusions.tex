Spin-Transfer Torque RAM (STT-RAM) is a promising candidate for future
on-chip cache design due to its high-density, low leakage, and immunity to
soft errors.  However, its high write latency and dynamic
write energy are the disadvantages compared to SRAM-based cache design.
In this paper, we propose to trade-off the non-volatility (data-retention time)
for better write performance/energy in STT-RAM cache design.
In this context, we conduct an application-driven study to characterize the
life time of a second level cache with the intention of using this time as the ideal
retention time for the STT-RAM. Execution-driven experiments with several PARSEC and SPEC benchmarks
indicate that at least 50\% of the cache blocks are updated in 10{\it ms} and thus, chose 10{\it ms}
as an optimal retention time by analyzing the STT-RAM retention time and write time trade-offs.
We investigate two design alternatives for avoiding the data loss due to the volatile nature of the
STT-RAM. The first approach writes back all the dirty blocks in the cache at the end of the retention time
and the second approach uses a buffering scheme to refresh the cache blocks that are not refreshed
during the retention time.

We analyze three different scenarios for designing the L2 cache: one with 1{\it sec} retention time with write back,
second with 10{\it ms} retention time with write back and the third with 10{\it ms} retention time with buffering, called
Revived-M-4MB. Compared to a base case design of 1MB per core SRAM design, the traditional non-volatile STT-RAM
cache with 4 times the SRAM capacity but high write latency, and the volatile STT-RAM with simple write back policy,
the proposed revive scheme shows both performance and power benefits across the application benchmarks studied in
this paper. The results not only indicate that it is possible to get up to 18\% improvement in speedup for PARSEC
benchmarks and 60\% reduction in total energy consumption over S-1MB design, but also show that the proposed design can be within 4\% of the ideal case with an equal capacity SRAM configuration, while being more energy efficient.
Furthermore, compared to the prior schemes that are aimed at hiding the high write latency of STT-RAMs,
the approach to reduce its write latency seems a better solution for designing a performance and power efficient
memory hierarchy for multi-cores.

%In conclusion, this paper demonstrates that it is possible to design a power and performance effective cache
%hiearchy by exploiting the tunable retention time of the emerging STT-RAMs.
