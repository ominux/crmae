Spin-Transfer Torque RAM (STT-RAM) is a promising candidate for future  
on-chip cache design due to its high-density, low leakage, and immunity to 
soft errors.  However, it's high write latency and dynamic
write energy are the disadvantages compared to SRAM-based cache design.
In this paper, we propose to trade-off the non-volatility (data retention time)
for better write performance/energy in STT-RAM cache design.
In this context, we conduct an application-driven study to characterize the
life time of a second level cache with the intention of using this time as the ideal
retention time for the STT-RAM. Execution-driven experiments with several PARASEC and SPEC benchmarks 
indicate that at leat 50\% of the cache blocka are updated in 10ms and thus, choose 10ms
as an optimal retention time by analyzing the SST-RAM retention time and write time trade-offs.
We investigate two design alternatives for avoiding the data lossdue to the volatile nature of the
STT-RAM. The first approach write backs all the dirty blocks in the cache at the end of the retention time
and the second approach uses a limited buffering scheme to refresh the cache blocks that are not refreshed 
during the retention time.

We analyze three different scenarios for designing the L2 cache: one with 1 second retention time with write back, 
second with 10ms retention time with write back and the third with 10ms retention time with buffering, called 
revived-STT-RAM. Compared to a base case design of 1MB per core SRAM design, the traditional non-volatile STT-RAM 
cache with 4 times the SRAM capqcity but high write latency, and the volatile STT-RAM with simple write back policy,
the proposed revive scheme shows both performance and power benefits across the application benchmarks studied in
this paper. The results not only indicate that it is possible to get up to XX\% improvement in instruction
throughput and YY\I reduction in total energy consumption, the proposed design can be within 4\% of the ideal case
with an equal capcity SRAM configuration, while being more energy efficient.
Furthermore, compared to the prior schemes that are aimed at hiding the high write latwency of STT-RAMs,
the approch to reduce its write latency seems a better solution for designing a performance and power efficient
memory hierarchy for multicores.
