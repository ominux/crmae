\begin{figure*} [t]
\centering
 \psfig{figure=figures/motiv.eps, width=6.25in, height=1.65in}
 \hrule
 \caption{\label{fig:motiv} \scriptsize \bf Caption.}
\end{figure*}

The ABSTRACT is to be in fully-justified italicized text. Use the word ``Abstract'' as the title, in
14-point Times, boldface type, centered relative to the column, initially capitalized. The abstract
is to be in 12-point, single-spaced type. Leave two blank lines after the Abstract, then begin the
main text. Look at previous MICRO abstracts to get a feel for style and length. referring
Fig~\ref{fig:motiv} here.

The ABSTRACT is to be in fully-justified italicized text. Use the word ``Abstract'' as the title, in
14-point Times, boldface type, centered relative to the column, initially capitalized. The abstract
is to be in 12-point, single-spaced type. Leave two blank lines after the Abstract, then begin the
main text. Look at previous MICRO abstracts to get a feel for style and length. referring
Fig~\ref{fig:motiv} here. The ABSTRACT is to be in fully-justified italicized text. Use the word
``Abstract'' as the title, in 14-point Times, boldface type, centered relative to the column,
initially capitalized. The abstract is to be in 12-point, single-spaced type. Leave two blank lines
after the Abstract, then begin the main text. Look at previous MICRO abstracts to get a feel for
style and length. referring Fig~\ref{fig:motiv} here. The ABSTRACT is to be in fully-justified
italicized text. Use the word ``Abstract'' as the title, in 14-point Times, boldface type, centered
relative to the column, initially capitalized. The abstract is to be in 12-point, single-spaced type.
Leave two blank lines after the Abstract, then begin the main text. Look at previous MICRO abstracts
to get a feel for style and length. referring Fig~\ref{fig:motiv} here. The ABSTRACT is to be in
fully-justified italicized text. Use the word ``Abstract'' as the title, in 14-point Times, boldface
type, centered relative to the column, initially capitalized. The abstract is to be in 12-point,
single-spaced type. Leave two blank lines after the Abstract, then begin the main text. Look at
previous MICRO abstracts to get a feel for style and length. referring Fig~\ref{fig:motiv} here. The
ABSTRACT is to be in fully-justified italicized text. Use the word ``Abstract'' as the title, in
14-point Times, boldface type, centered relative to the column, initially capitalized. The abstract
is to be in 12-point, single-spaced type. Leave two blank lines after the Abstract, then begin the
main text. Look at previous MICRO abstracts to get a feel for style and length. referring
Fig~\ref{fig:motiv} here. The ABSTRACT is to be in fully-justified italicized text. Use the word
``Abstract'' as the title, in 14-point Times, boldface type, centered relative to the column,
initially capitalized. The abstract is to be in 12-point, single-spaced type. Leave two blank lines
after the Abstract, then begin the main text. Look at previous MICRO abstracts to get a feel for
style and length. referring Fig~\ref{fig:motiv} here. The ABSTRACT is to be in fully-justified
italicized text. Use the word ``Abstract'' as the title, in 14-point Times, boldface type, centered
relative to the column, initially capitalized. The abstract is to be in 12-point, single-spaced type.
Leave two blank lines after the Abstract, then begin the main text. Look at previous MICRO abstracts
to get a feel for style and length. referring Fig~\ref{fig:motiv} here. The ABSTRACT is to be in
fully-justified italicized text. Use the word ``Abstract'' as the title, in 14-point Times, boldface
type, centered relative to the column, initially capitalized. The abstract is to be in 12-point,
single-spaced type. Leave two blank lines after the Abstract, then begin the main text. Look at
previous MICRO abstracts to get a feel for style and length. referring Fig~\ref{fig:motiv} here.

\begin{figure*} [t]
\centering
\begin{tabular}{cc}
 \psfig{figure=figures/motiv.eps, width=2.24in, height=1.0in} &
 \psfig{figure=figures/motiv.eps, width=2.24in, height=1.0in} \\
 \scriptsize (a) 4 fig-a & \scriptsize (b) 4 fig-a
\end{tabular}
 \hrule
 \caption{\scriptsize \bf Caption.}
\label{fig:2-placement}
\end{figure*}

The ABSTRACT is to be in fully-justified italicized text. Use the word ``Abstract'' as the title, in
14-point Times, boldface type, centered relative to the column, initially capitalized. The abstract
is to be in 12-point, single-spaced type. Leave two blank lines after the Abstract, then begin the
main text. Look at previous MICRO abstracts to get a feel for style and length. referring
Fig~\ref{fig:motiv} here. The ABSTRACT is to be in fully-justified italicized text. Use the word
``Abstract'' as the title, in 14-point Times, boldface type, centered relative to the column,
initially capitalized. The abstract is to be in 12-point, single-spaced type. Leave two blank lines
after the Abstract, then begin the main text. Look at previous MICRO abstracts to get a feel for
style and length. referring Fig~\ref{fig:motiv} here. The ABSTRACT is to be in fully-justified
italicized text. Use the word ``Abstract'' as the title, in 14-point Times, boldface type, centered
relative to the column, initially capitalized. The abstract is to be in 12-point, single-spaced type.
Leave two blank lines after the Abstract, then begin the main text. Look at previous MICRO abstracts
to get a feel for style and length. referring Fig~\ref{fig:motiv} here. The ABSTRACT is to be in
fully-justified italicized text. Use the word ``Abstract'' as the title, in 14-point Times, boldface
type, centered relative to the column, initially capitalized. The abstract is to be in 12-point,
single-spaced type. Leave two blank lines after the Abstract, then begin the main text. Look at
previous MICRO abstracts to get a feel for style and length. referring Fig~\ref{fig:motiv} here. The
ABSTRACT is to be in fully-justified italicized text. Use the word ``Abstract'' as the title, in
14-point Times, boldface type, centered relative to the column, initially capitalized. The abstract
is to be in 12-point, single-spaced type. Leave two blank lines after the Abstract, then begin the
main text. Look at previous MICRO abstracts to get a feel for style and length. referring
Fig~\ref{fig:motiv} here. The ABSTRACT is to be in fully-justified italicized text. Use the word
``Abstract'' as the title, in 14-point Times, boldface type, centered relative to the column,
initially capitalized. The abstract is to be in 12-point, single-spaced type. Leave two blank lines
after the Abstract, then begin the main text. Look at previous MICRO abstracts to get a feel for
style and length. referring Fig~\ref{fig:motiv} here. The ABSTRACT is to be in fully-justified
italicized text. Use the word ``Abstract'' as the title, in 14-point Times, boldface type, centered
relative to the column, initially capitalized. The abstract is to be in 12-point, single-spaced type.
Leave two blank lines after the Abstract, then begin the main text. Look at previous MICRO abstracts
to get a feel for style and length. referring Fig~\ref{fig:motiv} here. The ABSTRACT is to be in
fully-justified italicized text. Use the word ``Abstract'' as the title, in 14-point Times, boldface
type, centered relative to the column, initially capitalized. The abstract is to be in 12-point,
single-spaced type. Leave two blank lines after the Abstract, then begin the main text. Look at
previous MICRO abstracts to get a feel for style and length. referring Fig~\ref{fig:motiv} here.
referring Fig~\ref{fig:2-placement} here.


\begin{figure*}
\centering
\begin{tabular}{cc}
 \psfig{figure=figures/motiv.eps, width=2.24in, height=1.0in} &
 \psfig{figure=figures/motiv.eps, width=2.24in, height=1.0in} \\
 \scriptsize (a) 2 fig-a & \scriptsize (b) 2 fig-a \\
\end{tabular}
\begin{tabular}{cc}
 \psfig{figure=figures/motiv.eps, width=2.24in, height=1.0in} &
 \psfig{figure=figures/motiv.eps, width=2.24in, height=1.0in} \\
 \scriptsize (c) 2 fig-c & \scriptsize (d) 2 fig-d
\end{tabular}
 \hrule
 \caption{\scriptsize \bf Caption.}
\label{fig:4-placement}
\end{figure*}

The ABSTRACT is to be in fully-justified italicized text. Use the word ``Abstract'' as the title, in
14-point Times, boldface type, centered relative to the column, initially capitalized. The abstract
is to be in 12-point, single-spaced type. Leave two blank lines after the Abstract, then begin the
main text. Look at previous MICRO abstracts to get a feel for style and length. referring
Fig~\ref{fig:motiv} here. The ABSTRACT is to be in fully-justified italicized text. Use the word
``Abstract'' as the title, in 14-point Times, boldface type, centered relative to the column,
initially capitalized. The abstract is to be in 12-point, single-spaced type. Leave two blank lines
after the Abstract, then begin the main text. Look at previous MICRO abstracts to get a feel for
style and length. referring Fig~\ref{fig:motiv} here. The ABSTRACT is to be in fully-justified
italicized text. Use the word ``Abstract'' as the title, in 14-point Times, boldface type, centered
relative to the column, initially capitalized. The abstract is to be in 12-point, single-spaced type.
Leave two blank lines after the Abstract, then begin the main text. Look at previous MICRO abstracts
to get a feel for style and length. referring Fig~\ref{fig:motiv} here. The ABSTRACT is to be in
fully-justified italicized text. Use the word ``Abstract'' as the title, in 14-point Times, boldface
type, centered relative to the column, initially capitalized. The abstract is to be in 12-point,
single-spaced type. Leave two blank lines after the Abstract, then begin the main text. Look at
previous MICRO abstracts to get a feel for style and length. referring Fig~\ref{fig:motiv} here. The
ABSTRACT is to be in fully-justified italicized text. Use the word ``Abstract'' as the title, in
14-point Times, boldface type, centered relative to the column, initially capitalized. The abstract
is to be in 12-point, single-spaced type. Leave two blank lines after the Abstract, then begin the
main text. Look at previous MICRO abstracts to get a feel for style and length. referring
Fig~\ref{fig:motiv} here. The ABSTRACT is to be in fully-justified italicized text. Use the word
``Abstract'' as the title, in 14-point Times, boldface type, centered relative to the column,
initially capitalized. The abstract is to be in 12-point, single-spaced type. Leave two blank lines
after the Abstract, then begin the main text. Look at previous MICRO abstracts to get a feel for
style and length. referring Fig~\ref{fig:motiv} here. The ABSTRACT is to be in fully-justified
italicized text. Use the word ``Abstract'' as the title, in 14-point Times, boldface type, centered
relative to the column, initially capitalized. The abstract is to be in 12-point, single-spaced type.
Leave two blank lines after the Abstract, then begin the main text. Look at previous MICRO abstracts
to get a feel for style and length. referring Fig~\ref{fig:motiv} here. The ABSTRACT is to be in
fully-justified italicized text. Use the word ``Abstract'' as the title, in 14-point Times, boldface
type, centered relative to the column, initially capitalized. The abstract is to be in 12-point,
single-spaced type. Leave two blank lines after the Abstract, then begin the main text. Look at
previous MICRO abstracts to get a feel for style and length. referring Fig~\ref{fig:motiv} here.
referring Fig~\ref{fig:4-placement} here.

\begin{figure*} [t]
\begin{minipage}{0.50\textwidth}
\centering
 \psfig{figure=figures/motiv.eps, width=1.35in, height=1.35in}
\hrule \caption{\scriptsize \bf Caption.} \label{fig:1-fig}
\end{minipage}
\hfill
\begin{minipage}{0.50\textwidth}
\begin{tabular}{cc}
\psfig{figure=figures/motiv.eps, width=1.5in, height=1.35in} &
\psfig{figure=figures/motiv.eps, width=1.5in, height=1.35in} \\
\end{tabular}
 \hrule
 \caption{\scriptsize \bf Caption.}
\label{fig:2-fig}
\end{minipage}
\end{figure*}


The ABSTRACT is to be in fully-justified italicized text. Use the word ``Abstract'' as the title, in
14-point Times, boldface type, centered relative to the column, initially capitalized. The abstract
is to be in 12-point, single-spaced type. Leave two blank lines after the Abstract, then begin the
main text. Look at previous MICRO abstracts to get a feel for style and length. referring
Fig~\ref{fig:motiv} here. The ABSTRACT is to be in fully-justified italicized text. Use the word
``Abstract'' as the title, in 14-point Times, boldface type, centered relative to the column,
initially capitalized. The abstract is to be in 12-point, single-spaced type. Leave two blank lines
after the Abstract, then begin the main text. Look at previous MICRO abstracts to get a feel for
style and length. referring Fig~\ref{fig:motiv} here. The ABSTRACT is to be in fully-justified
italicized text. Use the word ``Abstract'' as the title, in 14-point Times, boldface type, centered
relative to the column, initially capitalized. The abstract is to be in 12-point, single-spaced type.
Leave two blank lines after the Abstract, then begin the main text. Look at previous MICRO abstracts
to get a feel for style and length. referring Fig~\ref{fig:motiv} here. The ABSTRACT is to be in
fully-justified italicized text. Use the word ``Abstract'' as the title, in 14-point Times, boldface
type, centered relative to the column, initially capitalized. The abstract is to be in 12-point,
single-spaced type. Leave two blank lines after the Abstract, then begin the main text. Look at
previous MICRO abstracts to get a feel for style and length. referring Fig~\ref{fig:motiv} here. The
ABSTRACT is to be in fully-justified italicized text. Use the word ``Abstract'' as the title, in
14-point Times, boldface type, centered relative to the column, initially capitalized. The abstract
is to be in 12-point, single-spaced type. Leave two blank lines after the Abstract, then begin the
main text. Look at previous MICRO abstracts to get a feel for style and length. referring
Fig~\ref{fig:motiv} here. The ABSTRACT is to be in fully-justified italicized text. Use the word
``Abstract'' as the title, in 14-point Times, boldface type, centered relative to the column,
initially capitalized. The abstract is to be in 12-point, single-spaced type. Leave two blank lines
after the Abstract, then begin the main text. Look at previous MICRO abstracts to get a feel for
style and length. referring Fig~\ref{fig:motiv} here. The ABSTRACT is to be in fully-justified
italicized text. Use the word ``Abstract'' as the title, in 14-point Times, boldface type, centered
relative to the column, initially capitalized. The abstract is to be in 12-point, single-spaced type.
Leave two blank lines after the Abstract, then begin the main text. Look at previous MICRO abstracts
to get a feel for style and length. referring Fig~\ref{fig:motiv} here. The ABSTRACT is to be in
fully-justified italicized text. Use the word ``Abstract'' as the title, in 14-point Times, boldface
type, centered relative to the column, initially capitalized. The abstract is to be in 12-point,
single-spaced type. Leave two blank lines after the Abstract, then begin the main text. Look at
previous MICRO abstracts to get a feel for style and length. referring Fig~\ref{fig:motiv} here.
referring Fig~\ref{fig:1-fig} and Fig.~\ref{fig:2-fig} here.
