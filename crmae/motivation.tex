


In order to utilize the volatile STT-RAM as the last level cache in designing an effective cache hierarchy,
we need to know what should be the ideal/feasible retention time. Ideally, the STT-RAM write  latency
should be competitive to SRAM  latency and the cache retention time should be  high.
However, as discussed in the following section, since the write latency is inversely propositional to the
retention time, we need to find a feasible tradeoff based on the STT-RAM device characteristics.
Thus, we first attempt to decide an ideal retention time by analyzing the characteristics of a last level
cache in a multiprogrammed environment. The idea is to understand the distribution of the inter-write 
interval and thus the average inter-write time to a last level cache and use this time as the STT-RAM
retention time.
This section describes our application-driven study to estimate the retention time.



\subsection{Relating Application Characteristics to Retention Time}

\begin{figure*} [t]
\centering
\begin{tabular}{cc}
 \psfig{figure=figures/parsec-hist.eps, width=3.4in, height=2.0in} &
\psfig{figure=figures/spec-hist.eps, width=3.4in, height=2.0in} \\
\scriptsize (a) PARSEC  & \scriptsize (b) SPEC 2006
\end{tabular}
 \hrule
 \caption{\scriptsize \bf Distribution of Blocks Showing Different Revival Times}
\label{fig:distribution}
\end{figure*}

Application characterization gives the basis for evaluating the impact of retention time on the overall system performance. In order to do this characterization, the first step is to  find an ideal time for which the cache block should retain the data.  A cache block is only refreshed when the block is written. Thus, we record intervals between two successive writes (refreshes) to the same L2 cache block. We define this interval to be {\it revival time}. While collecting these results, we ensure that if a block gets invalidated in between two consecutive writes, we don't consider the time in between the invalidation and the next write. Previous works ~\cite{brooks} do similar type of revival time analysis, but for L1 cache. Figure~\ref{fig:distribution} shows the distribution of L2 cache blocks having different revival time intervals. These results are obtained by running multi-threaded (PARSEC~\cite{PARSEC}) and multi-programmed (SPEC 2006~\cite{SPEC}) applications on the M5 Simulator~\cite{M5} that models a 2GHz processor consisting of 4 cores, with 4MB L2 cache. Table~\ref{table:sim_config} contains additional details of the system configuration. Figures~\ref{fig:distribution} (a) and (b) show the results of three PARSEC and SPEC benchmarks along with the averages across 12 PARSEC  and 14 SPEC benchmarks, respectively. We observe from the figure that, on an average, approximately 50\% of cache blocks get refreshed within 10 ms, this is in contrast to the microsecond reuse for L1 case~\cite{brooks}. About 20\% of blocks remain in the cache for more than 40 ms and rest of the blocks have intermediate revival times. We conclude that blocks which stay longer than the retention time in the cache without being refreshed are assumed to be not available, and would affect the application performance the most. This distribution also gives us the basis on which we can choose the optimal retention time. Reducing the retention time too much will make the cache too volatile leading to degraded performance, while increasing the retention time would affect the write latency. In the next subsection, we will see how increasing the retention time, has negative impacts on write latency. 



\subsection{Low Retention STT-RAM Characteristics}
Table~\ref{table:rt-wt} shows that there is significant reduction in write latency with reduction in retention time. Section 4 explains how these numbers originate. We want to clarify from device fabrication perspective that, these retention times are the most stable designs possible. As we lower the retention times of these STT-RAM cells in the range of {\it ms} it becomes much harder to precisely mark a STT-RAM cell with a fixed retention time. For the sake of correctness and preciseness we discuss these designs only in the paper. Later in the Section 7, it will be clear, that our design assumptions have no affect on the generality of the results.

\begin{table*}[t]
  \centering
  \caption{Retention and Write Latencies for STT-RAM L2 Cache}
  \label{table:rt-wt}
  \begin{tabular}{| c | c | c | c | c | c |}
  	\hline
	 Retention Time & 10years & 1sec &10ms \\
	\hline
	Write Latency (Latency Optimized) @2GHz & 22 cycles & 12 cycles & 6 cycles \\
	\hline
	Write Latency (Leakage Optimized)@2GHz & 10 cycles & 8 cycles & 8 cycles \\
	\hline
  \end{tabular}
\end{table*}

To analyze the tradeoffs between retention time and overall system performance, lets consider an utopian cache with 10 year retention time having minimum write latency and energy.  To bridge the gap between current and utopian cache, we need to reap the benefits of both: application characteristics and emerging device technology. From application side, it is best to choose a retention time which minimizes the number of unrefreshed blocks and from the technology side it is ideal to choose the STT-RAM with minimum write latency and energy. We choose 10 ms retention time as optimal retention time which balances both the sides. In Section 7, we do a sensitivity analysis by choosing retention times 100 ms, 500 ms and 1sec. In Section 5, we propose micro-architecture techniques to deal with blocks having revival time greater than 10ms. 




 
  


