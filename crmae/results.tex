In this section, we will discuss all the experimental results associated with our design choices.



\begin{figure*} [t]
%\begin{figure*} [t]
\centering
 \psfig{figure=figures/parsec-speedup.eps, width=6.9in, height=2.5in}
 \hrule
 \caption{\label{fig:parsec-new} \scriptsize \bf Normalized speedup for PARSEC Applications }
%\end{figure*}
\end{figure*}



\noindent\textbf {Performance:} Figure \ref{fig:parsec-speedup} shows speedup improvements of a subset of PARSEC 
multithreaded applications along with the average (taken across 12 PARSEC applications listed in Table~\ref{}). 
All speedup numbers are normalized to S-1MB. When M-4MB design is used, the applications to the right of x264 application (including x264) are
showing speedup improvements over S-1MB, inspite of high write latency of M-4MB.
The reason for this is multi-fold. These applications are benefiting from 4x capacity increase and also 
because of the presence of write buffers. To see the benefits of write buffer, 
consider fluidanimate and vips applications. Even though they have high number of writes to the L2 cache, 
they are benefiting from M-4MB because of the staggered writes which help write buffers in amleirorating increased write latency
of M-4MB.The applications to the left of x264 are write intensive applications. In all these applications we see degradation in speedup
because of the high write latency.  On an average, traditional 10 year STT-MRAM gives 3\% speedup improvement over S-1MB.



\begin{figure*} [t]
%\begin{figure*} [t]
\centering
 \psfig{figure=figures/spec-ws.eps, width=3.9in, height=2.5in}
 \hrule
 \caption{\label{fig:spec-new} \scriptsize \bf Normalized Average Instruction Throughput(IT) and Weighted Speedup(WS) for SPEC 2006 multiprogrammed mixes. }
%\end{figure*}
\end{figure*}


Consider Volatile M-4MB(1sec) design. This design has no refreshing scheme, but since 1 sec time 
covers almost all the refreshes as discussed in ~\ref{fig:distribution}, one can expect benefits of reduced write latency.
We observe benefits over M-4MB in all the applications. Volatile M-4MB(10ms) design also 
doesn't have any refreshing scheme but the retention time of STT-RAM
cells used is 10ms. Although this design has very less write latency, no refreshing scheme triggers
large number of write backs. For example in the case of vips, there is about 20\% speedup degradation over
M-4MB, because of the large number of write backs shown in~\ref{fig:writebacks}. It is interesting to see
the case of swaptions where there is slight improvement in speedup over M-4MB, although 
there is increase in number of writebacks. It is because of the reduced write latency and staggered write backs. 
Our scheme revived M-4MB(10ms), which incorporates refreshing scheme is showing speedup improvments
in almost all applications except fluidanimate. On an average our scheme is better than traditional 10yr
STT-RAM by 15\% and over Volatile STT-RAM (1sec) by 4.5\% (In case of facesim, our scheme is better than
Volatile STT-RAM (1 sec) by 22.7\%). Our scheme is closer to the ideal S-4M case, difference is 4\%.


Figure~\ref{fig:spec-new} shows instruction throughput and weighted speedup for multiprogram
mixes. We observe that our scheme revived M-4MB gives 22\% improvement in instruction throughput 
over M-4MB and 11\% improvement over Volatile STT-RAM (1 sec). (In case of the mix of bzip2, gcc, lbm, hmmer,
the improvement is 15\%). 
Weighted speedup improvement over M-4MB is 4\% and over Volatile STT-RAM(1 sec) is 2\%. 


\noindent\textbf {Energy:}


\begin{figure*} [t]
\centering
\begin{tabular}{c}
\psfig{figure=figures/legend.eps, width=5.5in, height=0.15in}
\end{tabular}
\begin{tabular}{ccc}
 \psfig{figure=figures/leak-eng.eps, width=2.1in, height=2.0in} &
\psfig{figure=figures/dyn-eng.eps, width=2.1in, height=2.0in} &
\psfig{figure=figures/tot-eng.eps, width=2.1in, height=2.0in} \\
\scriptsize (a) Leakage Energy  & \scriptsize (b) Dynamic Energy & \scriptsize (c) Total Energy
\end{tabular}
 \hrule
 \caption{\scriptsize \bf Energy of Applications Normalized to that of S-1MB}
\label{fig:energy}
\end{figure*}







\subsection{Sensitivity Analysis}

\begin{figure*} [t]
%\begin{figure*} [t]
\centering
 \psfig{figure=figures/writebacks.eps, width=4.9in, height=2.5in}
 \hrule
 \caption{\label{fig:writebacks} \scriptsize \bf Number of Write backs normalized to M-4MB}
%\end{figure*}
\end{figure*}


\begin{figure*} [t]
%\begin{figure*} [t]
\centering
 \psfig{figure=figures/confi.eps, width=3.9in, height=2.5in}
 \hrule
 \caption{\label{fig:confi} \scriptsize \bf Confidence Intervals of Dead Blocks for each Way}
%\end{figure*}
\end{figure*}

\noindent\textbf {Choosing correct buffer slots}

\noindent\textbf {Number of bits for counter}
There is no observable difference in performance and energy by increase in the number of bits of the counter. 
\subsection{Scaling Effects}







